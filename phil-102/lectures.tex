\documentclass{article}
    %%%%%%%%%%%%%%%%%%%%%%%%%%%%%%%%%%%%%%%%%%%%%%%%%%%%%%%%%%%%%
    % Lecture Specific Information to Fill Out
    %%%%%%%%%%%%%%%%%%%%%%%%%%%%%%%%%%%%%%%%%%%%%%%%%%%%%%%%%%%%%
    \newcommand{\LectureTitle}{Lecture Notes}
    \newcommand{\LectureDate}{\today}
    \newcommand{\LectureClassName}{Philosophy 102}
    \newcommand{\LatexerName}{Bernardo\ Meurer}
    %%%%%%%%%%%%%%%%%%%%%%%%%%%%%%%%%%%%%%%%%%%%%%%%%%%%%%%%%%%%%

    % Change "article" to "report" to get rid of page number on title page

    \usepackage[utf8]{inputenc}
    \usepackage{amsmath,amsfonts,amsthm,amssymb}
    \usepackage{setspace}
    \usepackage{Tabbing}
    \usepackage{fancyhdr}
    \usepackage{textcomp}
    \usepackage{lastpage}
    \usepackage{extramarks}
    \usepackage[bottom]{footmisc}
    \usepackage{chngpage}
    \usepackage{soul,color}
    \usepackage{graphicx,float,wrapfig}
    \usepackage{afterpage}
    \usepackage{abstract}

    % In case you need to adjust margins:
    \topmargin=-0.45in
    \evensidemargin=0in
    \oddsidemargin=0in
    \textwidth=6.5in
    \textheight=9.0in
    \headsep=0.25in

    % Setup the header and footer
    \pagestyle{fancy}
    \lhead{\LatexerName}
    \chead{\LectureClassName: \LectureTitle}
    \rhead{\LectureDate}
    %\lfoot{\lastxmark}
    \cfoot{}
    \rfoot{Page\ \thepage\ of\ \pageref{LastPage}}
    \renewcommand\headrulewidth{0.4pt}
    %\renewcommand\footrulewidth{0.4pt}
    %%%%%%%%%%%%%%%%%%%%%%%%%%%%%%%%%%%%%%%%%%%%%%%%%%%%%%%%%%%%%

    \begin{document}
    \begin{spacing}{1.2}
    \newpage
        \section*{Religion}
        \begin{itemize}
            \item ``Obligation, bond, reverence and respect for what is sacred.''
            \item ``That which binds or holds us together individually and collectively with awe and wonder in life.''
        \end{itemize}
        \subsection*{Elements of Religion}
        \begin{itemize}
            \item Functionalism --- That which relates to the formal structure and institutional aspect of religion.
            \item Essentialism --- \textinterrobang
            \item Belief --- That which is seen as fact, but is not proven; an acceptance that something is true without empirical proof.
            \item Sacred
            \item Secular --- That which is separate from religion
        \end{itemize}
        \subsection*{Types of Religion}
        \begin{itemize}
            \item Theism --- A belief that god exists.
            % FIXME: Double lines
            \item Atheism\footnote{\textit{Naturalism}} --- A belief that no god exists.
            \item Monotheism --- Belief in a singular god
            \item Polytheism --- Belief in many Gods/Goddesses
            \item Pantheism\footnote{\textit{Animism}} --- The universe is God, and God is the universe.
            \item Panentheism --- God is the universe, but it is also timelessly beyond the universe.
            \item Agnosticism --- A belief that the existance of God is unknowable.
            \item Deism --- A belief that one deity exists, it created the world, but it isn't involved.
            \item Humanism --- A belief that human beings alone are responsible for what happens on Earth. Usually bundled with a great belief in the, mostly innate, goodness of human beings.
        \end{itemize}
        \section{Primal, First, or Indigenous Religions}
        \subsection{Guest: Ernestine de Soto}
        \begin{itemize}
            \item Ernestine is a Chumash elder.
            \item Her mother was the last fluent speaking Chumash.
            \item The Chumash lived in the region from San Luis Obispo and
                under.
            \item Her lineage dates back to 13,000 years of living in this
                region.
            \item Recent evidence suggest that the Chumash were one of the very
                first groups of people to reach the Americas.
            \item Ernestine prefers animals to people.
        \end{itemize}
        \subsection{Myth}
    \begin{itemize}
        \item Myth --- Traditional story, especially one concerning the early
            history of a people or explaining natural or social phenomenon,
            typically involving supernatural events or beings.
    \end{itemize}
    \subsubsection{Navajo}
    \begin{itemize}
        \item The Navajo people believe in one all encompassing god, the Wind
    \end{itemize}
    \section{Hinduism}
    \begin{itemize}
        \item Lexicon
            \begin{itemize}
                \item Chandras --- Regular lay people
                \item Dana --- Charity / food for nuns and monks
                \item Digambara --- Sky clad
                \item Shevetambara --- White clad
                \item Jiva
                \item Ajiva
                \item Mukti --- Liberation
                \item Tirthankara
                \item Mahavira
                \item Jin
                \item Axial Age
                \item Kevala
                \item Five Principles --- Ahisma, Satya, Asteya, Brahmacharya,
                    Aparigraha.
            \end{itemize}
        \item Jains believe that each living thing has Jiva and Karma.
        \item Karma is like dust, that collects on the soul.
        \item Axial Agre 800 BCE --- 200 BCE was the era of the greek
            philosphers, Socrates and Plato with investigativity and democracy,
            The Chinese with Confucius and Lao Tzu, thinking of the humane and
            tao, of the hebrew prophets who talked about justice, morality, and
            community, as well as Buddha, with meditation and fighting the caste
            system.
        \item Mahavira taught that every living being has sanctity and dignity
            on its own accord. Everyone and everything was the same.
    \end{itemize}
    \section{Characteristic of Religion}
    \begin{itemize}
        \item Authority
        \item Ritual
        \item Speculation \& Explanation
        \item Tradition
        \item Grace
        \item Connection to the mystery of Life
    \end{itemize}
    ``When I know that the glass is already broken, then every minute with it is
    precious.''

    ``There is no knowledge without sacrifice.''

    ``In order to gain anything you must first lose everything.''

    \section{Buddhism}
    
    \end{spacing}
    \end{document}
