\documentclass[12pt,letterpaper]{article}
\usepackage[utf8]{inputenc}
\usepackage{setspace}
\usepackage[bottom]{footmisc}
\usepackage[margin=1in]{geometry}
\usepackage[notes,backend=biber]{biblatex-chicago}
\bibliography{paper-1}

\title{Rousseau, Locke, and Freedom: A \textsuperscript{\scriptsize{very}}Brief Survey of Political Philosophy in
the Enlightenment Era}
\author{Bernardo Meurer}
\date{\today}

\begin{document}
\maketitle
\newpage
\begin{spacing}{2}
    The age of enlightenment brought forth a lively debate surrounding freedom;
    what it means to be free, who deserves to be free, and why. At the center
    of this discussion stand Jean-Jacques Rousseau and John Locke, two of the
    most prolific political theorists of their era. Despite Rousseau's and
    Locke's views usually being pitted against each other, they are much more
    alike than one might recognize at first, in fact it is fair to say that
    Rousseau and Locke are two sides of the same coin; in other words, they
    simply have different approaches to the same problem.

    Contextually, it is important to understand what exactly the sides of our
    metaphorical coin are. The question of freedom is, I think, best represented
    as an eternal struggle between the interests of the individual, and the
    interests of the remaining public. In general, the search to define and
    outline the boundaries of freedom is congruent with searching for a balance
    between individual rights and the entitlements of the public.

    With the previous perspective in mind, we can start analyzing Rousseau's
    approach, which I consider to be public-first. The problem as described by
    Rousseau is ``To find a form of association that will defend and protect
    the person and goods of each associate with the full common force, and by
    means of which each, uniting with all, nevertheless
    obey only himself and remain as free as before.''\autocite[p.~49]{rousseau-1997}
    This association, as defined above, is Rousseau's most general definition of
    a legitimate civil order.\autocite[p.~xvii]{rousseau-1997} Under this order
    the associates pool all of their resources, material or not, and put
    themselves, and thus the society, under the guidance of its, and therefore
    their, ``general
    will.''\autocites[p.~xvii]{rousseau-1997}[p.~78]{perry-bock-2012} The
    problem of how to form and sustain said association, then, is what
    Rousseau's \textit{Social Contract} aims to
    solve.\autocite[p.~78]{perry-bock-2012}

    It may seem from Rousseau's statements in the previous paragraph, specially
    with regards to his idea of how one relates to society that he did not have
    a primary concern with Freedom, yet this could not be furthest from the
    truth. In Rousseau's view, it is precisely how ubiquitous and
    all-encompassing the Social Contract is that makes it's associates Free.
    This is best represented in the eighth paragraph of the first book of his
    essay \textit{Of The Social Contract}: ``Finally, each, by giving himself to
    all, gives himself to no one, and since there is no associate over whom one
    does not acquire the same right as one grants him over oneself, one gains
    the equivalent of all one loses, and more force to preserve what one
    has.''\autocite[p.~50]{rousseau-1997} This is to say that, assuming one is
    Free when his will is no more limited than that of any of his kind, then all
    of those who subscribe to the Social Contract are free within their
    \textit{people}.\footnote{As defined by Rousseau}

    I said previously that Rousseau's approach is public-first (as opposed to
    individual-first.) The reason for this is that Rousseau grants the sovereign
    public almost unlimited power over the citizens (as individuals.) Rousseau's
    framework of society simply does not accept the idea that an individual has
    rights that are beyond and/or against the
    state.\autocite[pp.~78--79]{perry-bock-2012} In Rousseau's words, ``if
    individuals were left some rights, then, since there would be no common
    superior who might adjudicate between them and the public, each, being
    judge in his own case on some issue, would soon claim to be so on all,
    the state of nature would subsist and the association necessarily become
    tyrannical or empty.''\autocite[p.~50]{rousseau-1997}

    Now, to the other side of the coin, we have Locke with his
    individual-first approach to Right. Even if orthogonal at surface, Locke's
    definition of property is an insight into his philosophy's \textit{Gestalt}:
    ``The Labor of one's Body and the Work of his Hands, we may say, are
    properly his.''\autocite[p.~287]{locke-1988} Property, then, comes not
    justified or validated by state or public, but rather by an individual's
    \textit{work}. This isn't to say that Locke had no concern for the public,
    he, for example,  limits appropriation to leaving ``enough, and as
    good''\autocite[p.~291]{locke-1988} of whatever resource is being
    appropriated to the remaining public. The key here is that for Locke it's
    one's individual will and labor that constitutes appropriation, with only
    minimal restrictions.\footnote{It is interesting here how Hegel will later
    follow a line similar to Locke's, but focus on it being an expression of
    one's personality, rather than just the result of their
    labor. \autocite[pp.~84--88]{hegel-1991}}

    The undertone of Lockean rights, then, is the individual; the government
    exists to preserve the individual's right to life, liberty, and
    property.\autocite[p.~63]{perry-bock-2012} This is so much the case that
    Locke, unlike Rousseau, gives the individuals right to dissolve the
    government, to rebel, and to stir revolution in case of a failing or inapt
    government. For Locke, the rights of the individual justify and require the
    rights and powers of the government, while for Rousseau we had the rights of
    the individual being defined by the requirements of the common public. There
    is, here, no ``General Will'', but rather solely the natural rights of Man
    and their interplay and interface give birth to government as it's
    institutional embodiment.\autocite{perry-bock-2012}

    If at first the statement that Rousseau and Locke were two sides of the same
    coin could have seemed obtuse, now it should be clear what was meant and
    what the statement entices. Fundamentally the difference solely lies in that
    Rousseau has a top-down approach, extrapolating the rights of individuals
    from the mechanisms and requirements of government, as defined in the Social
    Contract, while Locke has a bottom-up approach, yielding the powers of
    government from the rights and freedoms of individuals. While the difference
    may seem purely procedural, we see that it implies fundamental changes in
    the resulting interpretations of the workings of society, especially when it
    comes to the rights of individuals \textit{against} the government.

    While making claims as to which interpretation of freedom and government is
    better is outside the scope of this paper, it is important to take note of
    the inherent flaws that each system presents. In Rousseau's, the idea that
    there is an unquestionable, \textit{semper verum}, Will of the people seems
    to create an opening for the justification of
    totalitarianism.\autocite[pp.~78--79]{perry-bock-2012} Meanwhile, Locke's
    placing of the individual on-par with the common opens the system up for
    constant rebellion and instability.

    \textit{In summa}, both Locke and Rousseau tackle the issue of freedom in similar
    ways, yet they show us that thought and argument are not always subject to
    the commutativeness axiom, but nonetheless that serves only to enrich the
    discussion and broaden the pathways for us to attain a sustainable, free
    society.


\end{spacing}
\newpage
\printbibliography[heading=bibintoc,title={Bibliography}]
\end{document}
