\documentclass[12pt,letterpaper]{article}
\usepackage[utf8]{inputenc}
\usepackage{setspace}
\usepackage[bottom]{footmisc}
\usepackage[margin=1in]{geometry}
\usepackage[notes,backend=biber]{biblatex-chicago}
\bibliography{paper-2}

\title{Churchill and The Soviet Lebensraum}
\author{Bernardo Meurer}
\date{December 10, 2018}

\begin{document}
\maketitle
\newpage
\begin{spacing}{2}
    The sunsetting of the First World War left behind an overwhelming feeling
    of fear and awe among Europeans. The war had been a truly terrifying
    display of what the recent technological advances represented in the
    battlefield. On top of that, the relatively outdated strategic modes of
    battle caused the war to be even deadlier due to the dynamic between
    entrenchment and indirect fire. At home, the image of the returning
    soldiers, the internal and external scars they bore, and the overall fear
    that was felt would leave a deep mark in the population.

    It is not surprising, then, that once the war was over there was a
    tremendously strong will, specially among the victorious, to never face such
    tragedy again. Perhaps for the first time in many years there was a distinct
    coalition for peace in Europe, which would embody itself in the League of
    Nations. The crucial aspect of this, however, is that the embedded feeling
    within both governments and people wasn't that of aversion to war, for I
    doubt any reasonable individual would fault someone for that, but rather it
    was \emph{fear} of war.

    The fear of another conflict in the scale of the Great War begins to show
    itself during the very first months of conflict within the Weimar Republic.
    Despite being aware of Hitler's ideals and of the actions and \emph{Gestalt}
    of the Nazi party, most nations kept quiet so as not to stir conflict. With
    hindsight it is easy to see how it was the severity of the unconditional
    surrender, formalized in the Vienna treaty, that would set up the Weimar
    republic for failure and create the appropriate environment for Hitler's
    rise.

    After the initial conquest of power by the National Socialist Party, a
    quickly succession of violations of the terms of the Treaty of Vienna were
    made. By 1935 the German army had exceeded the number of troops stipulated
    by the treaty, as well as begun manufacturing airplanes. In the next year
    German soldiers would occupy the Rhineland, again infringing on the treaty.
    These movements culminated with the \emph{Anschluss}, the unification of
    Germany and Austria, and the occupation of the \emph{Sudetenland}. All these
    acts were, without a shadow of doubt, overwhelming proof that the Nazi
    government was not willing or wanting to keep the terms of the surrender.
    The response from the League of Nations, however, was a new one: a policy of
    appeasement. Fundamentally this involved avoiding to take any actions
    against Germany, and thus Hitler, that could cause new large-scale
    conflicts.

    While most voices in the European governments supported, in a silent
    unison, the appeasement, Winston Churchill loudly opposed it. While the
    League of Nations underestimated the imminent threat that Hitler was,
    Churchill managed to correctly understand what his rule of Germany enticed.
    Most importantly he was able to see the deep incompatibilities between the
    ``traditional'' Western values of democratic government, an indisputable
    fruit of the enlightenment, and the Fascism growing in Germany and Italy.
    Fundamentally, Churchill's brilliance here lays in his realizing that the
    coexistence of the systems and ideologies was unlikely, and that one was
    bound to overtake the other, especially given Hitler's constant focus on
    the volk's \emph{Lebensraum}.

    It is most probable that, beyond the ideological question, Churchill's
    alarmist approach regarding the Third Reich was propelled by his fear of the
    expansionist agenda that was inherent to the regime, again due to it's wish
    for an appropriate \emph{Lebensraum} for the Aryan people. It was clear for
    Churchill that this posed a threat not only to the British national
    territory but also to all of Europe, which most certainly where Hitler's
    primary land interests would lie.

    Given Churchill's foresight, and most importantly how unfortunately correct
    he was in his beliefs regarding Hitler, it should come as no surprise that
    as the Cold War geared up, he began to feel the same reprieves regarding the
    Soviet Union. After the end of the Second World War, it was clear that the
    USSR had expansionist tendencies, and crucially that the expansionism was
    not simply territorial, but also ideological.

    In his famous speech \emph{The Sinews of Peace}, Churchill states ``Nobody
    knows what Soviet Russia and its Communist international organisation
    intends to do in the immediate future, or what are the limits, if any, to
    their expansive and proselytising tendencies.'' No clearer state could be
    made regarding his views of the danger posed by the Soviet Union. In fact, I
    think perhaps he feared that the soviets had learned from the Nazi
    expansionist experiment and would be much harder to combat. In the speech
    he, for example, discusses the agreements made in the Yalta conference,
    which were without doubt a great diplomatic victory to the Soviets given
    their favorable terms.

    It is clear from the speech, as well as from other addresses given by
    Churchill at the time, that his greatest fear regarding the Soviet Union was
    for the Western nations to allow the same neglect that empowered Hitler's
    empire of evil to befall the USSR.\@ It was without a question a bitter
    taste to those that discredited Churchill in the days before the war to have
    seen what would unfold in between 1939 and 1945. Churchill, then, had a
    strong determination to do whatever was in his power to prevent the same to
    happen, and to ensure awareness of the danger the USSR posed to the West.

    In conclusion, there is no doubt that Churchill's experience with the Nazi
    menace, and in particular with the international neglect that allowed it to
    develop into a world-engulfing war, was a foundational experience and shaped
    his position and policy recommendations regarding the USSR.\@ In Communism,
    Churchill saw very similar ideological incompatibilities (to the west) as he
    did with Fascism, and in the Soviet Union he saw the same expansionism that
    had devastated Europe. It is no surprise that, given his experience, he
    believed that a united Europe, materialized in the European Union, and a
    painful awareness of the Soviet issue were the correct approach to ensure
    peace in the decades to come.
\end{spacing}
\newpage
%\printbibliography[heading=bibintoc,title={Bibliography}]
\end{document}
