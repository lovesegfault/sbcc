\documentclass{article}
%%%%%%%%%%%%%%%%%%%%%%%%%%%%%%%%%%%%%%%%%%%%%%%%%%%%%%%%%%%%%
% Lecture Specific Information to Fill Out
%%%%%%%%%%%%%%%%%%%%%%%%%%%%%%%%%%%%%%%%%%%%%%%%%%%%%%%%%%%%%
\newcommand{\LectureTitle}{Lecture Notes}
\newcommand{\LectureDate}{\today}
\newcommand{\LectureClassName}{History 104}
\newcommand{\LatexerName}{Bernardo\ Meurer}
%%%%%%%%%%%%%%%%%%%%%%%%%%%%%%%%%%%%%%%%%%%%%%%%%%%%%%%%%%%%%

% Change "article" to "report" to get rid of page number on title page

\usepackage[utf8]{inputenc}
\usepackage{amsmath,amsfonts,amsthm,amssymb}
\usepackage{setspace}
\usepackage{Tabbing}
\usepackage{fancyhdr}
\usepackage{lastpage}
\usepackage{todonotes}
\usepackage{extramarks}
\usepackage{chngpage}
\usepackage{soul,color}
\usepackage{graphicx,float,wrapfig}
\usepackage{afterpage}
\usepackage{abstract}

% In case you need to adjust margins:
\topmargin=-0.45in
\evensidemargin=0in
\oddsidemargin=0in
\textwidth=6.5in
\textheight=9.0in
\headsep=0.25in

% Setup the header and footer
\pagestyle{fancy}
\lhead{\LatexerName}
\chead{\LectureClassName: \LectureTitle}
\rhead{\LectureDate}
\lfoot{\lastxmark}
\cfoot{}
\rfoot{Page\ \thepage\ of\ \pageref{LastPage}}
\renewcommand\headrulewidth{0.4pt}
\renewcommand\footrulewidth{0.4pt}

%%%%%%%%%%%%%%%%%%%%%%%%%%%%%%%%%%%%%%%%%%%%%%%%%%%%%%%%%%%%%

\begin{document}
\begin{spacing}{1.2}
    \newpage
    \section{Introduction}
    \subsubsection*{The Pre-Modern West}
    \begin{enumerate}
        \item Catholic Christian
        \item Greco-Roman influence
        \item Germanic language and culture
    \end{enumerate}
    \subsection*{The Modern West}
    \begin{enumerate}
        \item Advanced technology
        \item Importance of science
        \item Immense wealth
        \item Liberal principles
        \item Religious freedom
    \end{enumerate}
    \section{Modernizing Europe}
    \begin{itemize}
        \item Even though there is no separation between church and state they
            are identifiable institutions.
        \item It makes no sense to say ``this is solely a religious issue, not a
            political one''; Rebellion against the king is a religious and political
            crime.
        \item The pope is, more formerly, the Bishop of Rome
        \item Peter, appointed by Jesus, is the first Bishop of Rome. With this,
            subsequent popes are considered to have the same power and importance
            as Peter.
        \item Even after the foundation of the Vatican, the pope continues to
            be the Bishop of Rome. It is considered shameful, and a crisis for a
            Pope to not live in Rome.
        \item The Church tries and condemns heretics, but it is the king that
            actually burns them alive
        \item The firs truly successful and thoroughly impacting heretic was
            Martin Luther
        \item One of the reasons Luther was so successful was due to the fact
            that he was so deeply and recognizably catholic. He was a monk, which
            was seen as a duty only to those most dedicated to the faith, and he
            had obtained a doctorate in technology
        \item Another important factor is that Luther came on the scene at a
            moment where
        \item Even though it was possible for a pope to denounce a king, putting
            in question his divine right, it was not popular, and in the Modern
            period, not very effective.
        \item Due to the rise of Protestantism wars break out. Between Catholics
            and Protestants and between Protestants and Protestants.
    \end{itemize}
    \subsection{The Defenestration of Prague}
    \begin{itemize}
        \item The Thirty years war goes from 1618 to 1648, and it was mainly
            fought in Germany, which was called the Holy Roman Empire
        \item Quite seriously, the only reason it was called the Holy Roman
            Empire was because it sounded cool. As the joke goes, it was not Holy,
            nor Roman, nor an Empire; and that has some truth to it.
        \item The war started with the Defenestration of Prague, which was an
            event where Catholics were thrown out of a window. (Check)
        \item The war was brutal, some towns lost \(\frac{2}{3}\) of their
            populations
        \item Rubens' \textit{Consequences of War} shows many Europeans were becoming turned off by the excesses of religious zeal.
        \item This was made after the 30 years war, and it was a painting protesting war. The new attitude after the war is that it was not worth it to fight over religious differences.
    \end{itemize}
    \subsection{The Rebirth of Science}
    \begin{itemize}
        \item Aristotle was known as ``The Philosopher''
        \item Aristotle was in part very famous because the Catholic church
            had, for centuries, backed Aristotle's views as correct.
        \item Ptolemy adopted and helped popularize Aristotle's ideology.
        \item Aristotle and Ptolemy proposed a geocentric universe surrounded by
            concentric spheres.
        \item The outermost sphere was called the ``Prime Mover'', and it's
            movement is what causes all the other spheres to move.\todo{Insert
            image.}
        \item Aristotle and Ptolemy spoke of ``Five Natural Elements''. The four
            elements that can be found on earth are water, earth, air and fire.
        \item The fifth element was called \textbf{Quintessence}, which later on
            would be called ``Ether'', and it was considered to be the perfect
            element. All eternal things, such as the moon and the sun, were
            thought to be made of quintessence due to their eternality.
        \item If you observe the movement of the planets against the stars you
            see that some of the planets don't seem to move consistently.
            Sometimes it seems as if the planet will start moving
            \textit{backwards} before it starts moving forwards again. This is
            called \textbf{Retrograde motion}.
        \item The Catholic Church used ``Epicycles'', which were smaller spheres
            embedded into the larger Aristotelian spheres, to explain the
            retrograde motion.\todo{Insert Image}
        \item The Epicycle was heavily criticized for violating Occam's Razor.
        \item As a simpler solution to the problem, Copernicus proposed a
            heliocentric universe.
        \item Even under a pseudonym it was dangerous for Copernicus to publish
            his ideas. At the time, in order to publish anything the approval of
            the local Bishop was required.
        \item Later on, after Copernicus' death, Galileo went on to promote his
            views posthumously.
        \item Furthermore Galileo disproved the existence of the quintessence
            with his telescope by observing the Sun and the Moon and seeing
            their inherit physical flaws.
        \item Galileo was tried for heresy, and after admitting his views were
            wrong and commit to no longer attempt to spread them, condemned to
            house arrest for the rest of his life.
        \item In his trial, Galileo did not argue that the Church was
            \textit{wrong}, but rather that it had misinterpreted sections of
            the Bible, by taken them more literally than they were intended to.
        \item In a certain sense \textbf{the Church had won the battle, but lost
            the war}.
        \item After Galileo Kepler, Ticho Brahe's apprentice, showed that
            planets move in elliptical and not circular motion, and thus the
            proposed concentric spheres cannot exist.
        \item Newton emphasized the similarity of causes for everything in the
            universe through four rules
            \begin{enumerate}
                \item \textit{We are to admit no more causes of natural things than such
                        as are both true and sufficient to explain their
                    appearances.}
                \item \textit{Therefore to the same natural effects we must, as far as
                    possible, assign the same causes.}
                \item \textit{The qualities of bodies, which admit neither
                        intensification nor remission of degrees, and which are
                        found to belong to all bodies within the reach of our
                        experiments, are to be esteemed the universal qualities of
                    all bodies whatsoever.}
                \item \textit{In experimental philosophy we are to look upon
                        propositions inferred by general induction from phenomena
                        as accurately or very nearly true, not withstanding any
                        contrary hypothesis that may be imagined, till such time as
                        other phenomena occur, by which they may either be made
                    more accurate, or liable to exceptions.}
            \end{enumerate}
        \item Francis Bacon is representative of how early modern ``scientists''
            rejected reliance on ancient authorities such as Aristotle and
            Ptolemy.
    \end{itemize}
    \section{Absolutism vs. Constitutionalism}
    \begin{itemize}
        \item Unintuitively, during the 17th century the kings seek to have more
            power, they wish to become absolutists.
        \item Examples of absolutist kings: James I of England, Louis XIV of
            France.
        \item Absolutism means ``No one can say no to the king.''
        \item Medieval kings had granted privileges to three important groups to
            keep them loyal
            \begin{itemize}
                \item Nobility
                \item Clergy
                \item Rich townspeople
            \end{itemize}
        \item These three groups together accounted to no more than 10\% of the
            population.
        \item Two medieval ``Privileges'', or ``Liberties'' were given to those
            groups
            \begin{itemize}
                \item Local autonomy --- The right to manage your own land
                    without interference from the King.
                \item Right of consultation --- The right to be consulted by the
                    king in cases where the king needs to make difficult
                    decisions.
            \end{itemize}
        \item With absolutism, kings seek to abolish that system, and remove
            these privileges from the nobility, clergy, and the rich.
        \item One of the reasons for this wish for absolute power was the desire
            to prevent religious civil wars.
        \item The other reason for the wish for more power was that kings aimed
            to raise taxes to cover the rising costs of war and government.
        \item Louis XIV was the most powerful absolutist king in 17th century
            Europe.\todo{Insert image (?)}
        \item Louis XIV's Palace of Versailles, where a lot of the nobility
            lived, was partially a huge power move.
        \item The English monarchy had a strong tradition of cooperation with
            Parliament.
        \item James II was openly Catholic (as opposed to Protestant) and
            absolutist. He was force to flee England during the Glorious
            Revolution of 1688.
        \item William and Mary, his daughter, became joint rulers of England
            after signing the Declaration of Rights, which prohibited them from
            becoming absolutist rulers.
        \item The Declaration of Rights states that there can be no sitting king
            of England that is not a Protestant.
        \item John Locke that the ruled have a right to remove a tyrant.
    \end{itemize}
    \section{The French Revolution}
    \begin{itemize}
        \item The French Revolution popularized the ideals of liberty, equality,
            and fraternity throughout Europe.
        \item Louis XIV's financial problems forced him to call the Estates
            General in 1789.
        \item The Estates general consisted of the Church, the Nobility, and the
            commoners.
        \item The Third Estate demanded double representation in 1789 and then
            later individual voting.
        \item The Third Estate declared itself a National Assembly and promised
            to give France its first written constitution.
        \item The Storming of Bastille
        \item The Declaration of the Rights of Man and of Citizens was a statement of the
            revolution's principles.
        \item Nowhere in the Declaration there was mention of Democracy.
        \item Louis rejected to sign the declaration, but a Parisian mob
            intimidated Louis XVI to move back to Paris and to accept the
            Declaration of the Rights of Man and of Citizens.
        \item The National Assembly completed the new constitution in 1791, but
            it was very controversial.
        \item 1789 --- 1792: National Assembly / Constitutional Monarchy
        \item 1793 --- 1794: National Convention / Republican
            Democracy (Robespierre)
        \item 1795 --- 1798: Chaos
        \item 1799 --- 1815: Napoleon
    \end{itemize}
    \section{Napoleon}
    \begin{itemize}
        \item Napoleon reestablished order in France at the expense of many
            revolutionary freedoms.
        \item When Napoleon seizes power he makes himself First Consul, and
            enacts a new constitution of France, allowing for all rich men to
            vote.
        \item Napoleon claimed ``I am the revolution.''
        \item The Napoleonic Code protected several revolutionary principles
            such at religious freedoms, property rights, and legal equality.
        \item Napoleon married into the Hapsburg family in 1810 to provide
            himself an heir.
    \end{itemize}
\end{spacing}
\end{document}
