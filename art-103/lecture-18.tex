\documentclass{article}
    %%%%%%%%%%%%%%%%%%%%%%%%%%%%%%%%%%%%%%%%%%%%%%%%%%%%%%%%%%%%%
    % Lecture Specific Information to Fill Out
    %%%%%%%%%%%%%%%%%%%%%%%%%%%%%%%%%%%%%%%%%%%%%%%%%%%%%%%%%%%%%
    \newcommand{\LectureTitle}{Lecture \#18 Notes}
    %\newcommand{\LectureDate}{\today}
    \newcommand{\LectureDate}{October\ 30,\ 2017}
    \newcommand{\LectureClassName}{ART\ 103}
    \newcommand{\LatexerName}{Bernardo\ Meurer}
    %%%%%%%%%%%%%%%%%%%%%%%%%%%%%%%%%%%%%%%%%%%%%%%%%%%%%%%%%%%%%
    
    % Change "article" to "report" to get rid of page number on title page
    \usepackage{amsmath,amsfonts,amsthm,amssymb}
    \usepackage{setspace}
    \usepackage{Tabbing}
    \usepackage{fancyhdr}
    \usepackage{lastpage}
    \usepackage{extramarks}
    \usepackage{chngpage}
    \usepackage{soul,color}
    \usepackage{graphicx,float,wrapfig}
    \usepackage{afterpage}
    \usepackage{abstract}
    
    % In case you need to adjust margins:
    \topmargin=-0.45in
    \evensidemargin=0in
    \oddsidemargin=0in
    \textwidth=6.5in
    \textheight=9.0in
    \headsep=0.25in
    
    % Setup the header and footer
    \pagestyle{fancy}
    \lhead{\LatexerName}
    \chead{\LectureClassName: \LectureTitle}
    \rhead{\LectureDate}
    \lfoot{\lastxmark}
    \cfoot{}
    \rfoot{Page\ \thepage\ of\ \pageref{LastPage}}
    \renewcommand\headrulewidth{0.4pt}
    \renewcommand\footrulewidth{0.4pt}

        %%%%%%%%%%%%%%%%%%%%%%%%%%%%%%%%%%%%%%%%%%%%%%%%%%%%%%%%%%%%%
    % Some tools
    \newcommand{\enterTopicHeader}[1]{\nobreak\extramarks{#1}{#1 continued on next page\ldots}\nobreak%
    \nobreak\extramarks{#1 (continued)}{#1 continued on next page\ldots}\nobreak}
\newcommand{\exitTopicHeader}[1]{\nobreak\extramarks{#1 (continued)}{#1 continued on next page\ldots}\nobreak%
   \nobreak\extramarks{#1}{}\nobreak}

\newlength{\labelLength}
\newcommand{\labelAnswer}[2]
{\settowidth{\labelLength}{#1}
\addtolength{\labelLength}{0.25in}
\changetext{}{-\labelLength}{}{}{}
\noindent\fbox{\begin{minipage}[c]{\columnwidth}#2\end{minipage}}
\marginpar{\fbox{#1}}

% We put the blank space above in order to make sure this
% \marginpar gets correctly placed.
\changetext{}{+\labelLength}{}{}{}}

\setcounter{secnumdepth}{0}
\newcommand{\TopicName}{}
\newcounter{TopicCounter}
\newenvironment{Topic}[1][Problem \arabic{TopicCounter}]
{\stepcounter{TopicCounter}
\renewcommand{\TopicName}{#1}
\section{\TopicName}
\enterTopicHeader{\TopicName}}
{\exitTopicHeader{\TopicName}}

\setcounter{secnumdepth}{0}
\newcommand{\ExampleSectionName}{}
\newcounter{ExampleSectionCounter}[TopicCounter]
\newenvironment{ExampleSection}[1][Example \arabic{ExampleSectionCounter}]
{\stepcounter{ExampleSectionCounter}
\renewcommand{\ExampleSectionName}{#1}
\section{\ExampleSectionName}
\enterTopicHeader{\ExampleSectionName}}
{\exitTopicHeader{\ExampleSectionName}}

\setcounter{secnumdepth}{0}
\newcounter{ExampleBoxCounter}[TopicCounter]
\newcommand{\examplebox}[1]
{
% We put this space here to make sure we're disconnected from the previous
% passage
\stepcounter{ExampleBoxCounter}
\noindent\fbox{\begin{minipage}[c]{\columnwidth}#1\end{minipage}}\enterTopicHeader{\ExampleSectionName}\exitTopicHeader{\ExampleSectionName}\marginpar{\fbox{\#\arabic{ExampleBoxCounter}}}
% We put the blank space above in order to make sure this
% \marginpar gets correctly placed.
\vskip10pt%
}

\renewcommand{\contentsname}{{\normalsize Topics Covered}}
\renewcommand{\abstractname}{\LectureTitle\ Summary}
\renewcommand{\absnamepos}{flushleft}

%%%%%%%%%%%%%%%%%%%%%%%%%%%%%%%%%%%%%%%%%%%%%%%%%%%%%%%%%%%%%
\begin{document}
    \begin{spacing}{1.2}
    \newpage
        \section{Late Antiquity}
        \subsection{Dura-Europos}

        (8 --- 1) They put decidedly Christian motifs in the sarcophagus, it has scenes from the old and new testaments. They have not, however, come up with a stable way to depict these scenes.
        
        (8 --- 5) Later, when they start to build Christian churches, they do it over the catacombs. They see it as an act of worship, respect, to their martyrs.
        
        (8 --- 6) Here it is very clear from the ceiling the influences from Rome, as we had seen in Ostia.

        (8 --- 7) We don't see a lot of sculptures from this time, mostly we see reliefs and frescos or wall paintings. 

        (8 --- 8) Here we see Christ ad Good Shepherd, however he is wearing a Roman toga and roman sandals. 

        (8 --- 9) Old Saint Peters was built where they believed that Peter had been buried. 

        (8 --- 11) This is the Basilica of Santa Constanza, it is different from churches we previously saw due to it being circular (like the Roman circular temples).

        (8 --- 15) We don't have a lot of 2-dimensional pictorial works from this time. 

        (8 --- 16) Here again we see Christ as a young, beardless, man. He also seems to be impervious to pain even while being crucified. 

        (8 --- 18) This relates to Saint Peters because you have this very long view, and it is showing that they are starting to set up a program, a standard way to do these churches.

    \end{spacing}
\end{document}