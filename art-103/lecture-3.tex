\documentclass{article}
    %%%%%%%%%%%%%%%%%%%%%%%%%%%%%%%%%%%%%%%%%%%%%%%%%%%%%%%%%%%%%
    % Lecture Specific Information to Fill Out
    %%%%%%%%%%%%%%%%%%%%%%%%%%%%%%%%%%%%%%%%%%%%%%%%%%%%%%%%%%%%%
    \newcommand{\LectureTitle}{Lecture \#3 Notes}
    %\newcommand{\LectureDate}{\today}
    \newcommand{\LectureDate}{August\ 30,\ 2017}
    \newcommand{\LectureClassName}{ART\ 103}
    \newcommand{\LatexerName}{Bernardo\ Meurer}
    %%%%%%%%%%%%%%%%%%%%%%%%%%%%%%%%%%%%%%%%%%%%%%%%%%%%%%%%%%%%%
    
    % Change "article" to "report" to get rid of page number on title page
    \usepackage{amsmath,amsfonts,amsthm,amssymb}
    \usepackage{setspace}
    \usepackage{Tabbing}
    \usepackage{fancyhdr}
    \usepackage{lastpage}
    \usepackage{extramarks}
    \usepackage{chngpage}
    \usepackage{soul,color}
    \usepackage{graphicx,float,wrapfig}
    \usepackage{afterpage}
    \usepackage{abstract}
    
    % In case you need to adjust margins:
    \topmargin=-0.45in
    \evensidemargin=0in
    \oddsidemargin=0in
    \textwidth=6.5in
    \textheight=9.0in
    \headsep=0.25in
    
    % Setup the header and footer
    \pagestyle{fancy}
    \lhead{\LatexerName}
    \chead{\LectureClassName: \LectureTitle}
    \rhead{\LectureDate}
    \lfoot{\lastxmark}
    \cfoot{}
    \rfoot{Page\ \thepage\ of\ \pageref{LastPage}}
    \renewcommand\headrulewidth{0.4pt}
    \renewcommand\footrulewidth{0.4pt}
    
    %%%%%%%%%%%%%%%%%%%%%%%%%%%%%%%%%%%%%%%%%%%%%%%%%%%%%%%%%%%%%
    % Some tools
    \newcommand{\enterTopicHeader}[1]{\nobreak\extramarks{#1}{#1 continued on next page\ldots}\nobreak%
                                        \nobreak\extramarks{#1 (continued)}{#1 continued on next page\ldots}\nobreak}
    \newcommand{\exitTopicHeader}[1]{\nobreak\extramarks{#1 (continued)}{#1 continued on next page\ldots}\nobreak%
                                       \nobreak\extramarks{#1}{}\nobreak}
    
    \newlength{\labelLength}
    \newcommand{\labelAnswer}[2]
      {\settowidth{\labelLength}{#1}
       \addtolength{\labelLength}{0.25in}
       \changetext{}{-\labelLength}{}{}{}
       \noindent\fbox{\begin{minipage}[c]{\columnwidth}#2\end{minipage}}
       \marginpar{\fbox{#1}}
    
       % We put the blank space above in order to make sure this
       % \marginpar gets correctly placed.
       \changetext{}{+\labelLength}{}{}{}}
    
    \setcounter{secnumdepth}{0}
    \newcommand{\TopicName}{}
    \newcounter{TopicCounter}
    \newenvironment{Topic}[1][Problem \arabic{TopicCounter}]
      {\stepcounter{TopicCounter}
       \renewcommand{\TopicName}{#1}
       \section{\TopicName}
       \enterTopicHeader{\TopicName}}
      {\exitTopicHeader{\TopicName}}
    
    \setcounter{secnumdepth}{0}
    \newcommand{\ExampleSectionName}{}
    \newcounter{ExampleSectionCounter}[TopicCounter]
    \newenvironment{ExampleSection}[1][Example \arabic{ExampleSectionCounter}]
      {\stepcounter{ExampleSectionCounter}
       \renewcommand{\ExampleSectionName}{#1}
       \section{\ExampleSectionName}
       \enterTopicHeader{\ExampleSectionName}}
      {\exitTopicHeader{\ExampleSectionName}}
    
    \setcounter{secnumdepth}{0}
    \newcounter{ExampleBoxCounter}[TopicCounter]
    \newcommand{\examplebox}[1]
      {
      % We put this space here to make sure we're disconnected from the previous
       % passage
       \stepcounter{ExampleBoxCounter}
       \noindent\fbox{\begin{minipage}[c]{\columnwidth}#1\end{minipage}}\enterTopicHeader{\ExampleSectionName}\exitTopicHeader{\ExampleSectionName}\marginpar{\fbox{\#\arabic{ExampleBoxCounter}}}
       % We put the blank space above in order to make sure this
       % \marginpar gets correctly placed.
       \vskip10pt%
       }
    
    \renewcommand{\contentsname}{{\normalsize Topics Covered}}
    \renewcommand{\abstractname}{\LectureTitle\ Summary}
    \renewcommand{\absnamepos}{flushleft}
    
    %%%%%%%%%%%%%%%%%%%%%%%%%%%%%%%%%%%%%%%%%%%%%%%%%%%%%%%%%%%%%
    
    \begin{document}
    \begin{spacing}{1.2}
    \newpage
    \section{Ancient Mesopotamia and Persia}
    Some of the first records we have of the are are pictographs. At first, 
    simple drawings of what wanted to be depicted, and later on following some
    rules. Their pictographs were later standardized by using 5 marks with a
    stylus.

    The White Temple (2--2) is a large temple in the area. It is a large,
    elevated construction, probably to make them closer to whatever they
    believed to be up in the sky. Curiously, the four corners of the White 
    Temple are aligned with the 4 cardinal directions. We believe the temple
    to be dedicated to Anu, the sky god. The tallest \emph{ziggurat} of all
    was known to the hebrews as The Tower of Babylon.
    
    (2--4) We believe this head to be a representation of the Goddess Inanna.

    (2--1) We believe this vase to be used in presenting offerings to the Gods.
    It was found in the same temple as (2--4), and depicts images of rituals 
    around it. We believe the rituals were to thank Innana for the abundance
    she provides. We refer to the horizontal ``lines'' as \emph{registers},
    they divide different sections of information in the work.
    
    (2--6) The Samaritans were one of the first to place figures like these in
    their shrines. The large eyes are idiosyncratic of their statues, and 
    symbolize their wakefulness, watchfulness, for the Gods. Some of these 
    bear inscriptions with the name of the donor, and sometimes even specific
    prayers for the Gods.
    
    (2--6) We call this the Stele of the Vultures. It celebrates the victory of 
    Eannanatum from Girsu (modern Telloh) over the neighboring cities. The
    number of spears, even if completely incompatible with the number of heads,
    is an attempt at representing the size of the army. Ont he right of the
    picture on can see Eannanatum commanding his army, and underneath the army
    the trampled bodies of their enemies.

    (2--7) This is the Standard of Ur. There is some debate still on whether
    this is actually a war banner, some believe it to be the casing for a
    music box.On the war side, in the top register we can see what is believe
    to be some kind of ruler, the larger image in the center. On the peace
    side, we can see a feast, someone playing the lyre, people carrying
    provisions, and symbols of plentifulness.

    (2--9, 2--10) The registered display of this work (2--10) is familiar, and
    reminds us of the previous works. This is one of the earliest depictions
    of \emph{composite} figures, mythical creatures that mix humans and animals.
    In the top register we see a \emph{Heraldic device} (a human form with two
    animals flanking it), this type of depiction will transcend civilizations,
    and this is one of the first ones we are aware of.
    
    (2--11) Cylinder seals are carved stone cylinders that, when rolled over a
    soft surface, would imprint it's drawings on it. These signified high
    positions, you had to be a ruler or someone of important to own a seal

    \end{spacing}
    \end{document}