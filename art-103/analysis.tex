\documentclass[12pt,letterpaper]{article}
    \usepackage[utf8]{inputenc}
    \usepackage{ifpdf}
    \usepackage{mla}
    \usepackage{todonotes}
    \usepackage[hidelinks]{hyperref}
    
    \urlstyle{same}
    
    \begin{document}
    \begin{mla}{Bernardo}{Meurer}{Dr. Villander}{Art 103}{November 8, 2017}%
        {On ``The museum as town hall''}
    %%%% Define the problem
        In ``The museum as town hall'', Bruce Cole tackles the recent push in transitioning museums from their traditional function (to curate, research, display, and judge works of art) to a more (supposedly) modern approach of having he museum as a ``town hall'', a space where citizens can debate and analyze issues of their communities. The problem with this is, as Cole points out, that this is simply \emph{not} the function of museums at all, in fact, we have a structure with that function, and it is the town hall itself. This (typically post-modernist) volition to repurpose spaces and change functions, shows here its ultimate weakness: the fact that eventually you're making things less efficient, not more. In the end, were we to bow down to trends such as this, we would end up with a town hall with its own socio-political functions, and museums having the same functions, and possibly another structure taking the functions previously associated with museums, effectively only shuffling things around. Anyone should, by now, already see an issue with this idea, namely that if people go to museums (and we know that to be true), then changing what museums do would simply create space for something else to take their old functions, effectively changing nothing. Moreover, how does Art come in on the town-hall-ness of the post-modern museum is unclear at best, which is troubling if we take museums as being temples for and of art.

        This sort of problem, post-modernist trend-cult attempting to change and remold everything just for the sake of it, is in no way, however, limited to the Arts. As a computer scientist, this cultural idea is abundantly clear. Every quarter a new, shiny, piece of technology is released in the software development world and, without fault, every quarter all the developers jump ship on their current methodology and flock to the hot new one. This behavior has been dubbed ``hype-oriented programming'' and, although not exactly the same as the one observed in museum directors, the similarity lies in that the trend, and the means through which the final product is created, become more important than the product itself, in machiavellian terms, the means justify the ends. It no longer matters whether our exhibition is coherent and of quality, what matters is that we curated it through instagram; or it does not matter that my note-taking app takes over a minute to load, it was built with the latest framework that everyone is talking about. Fundamentally they are both the same issue, and this same line of thinking can be seen in a variety of other fields throughout both the Sciences and the Humanities.

    %%%% What's the evidence
        One of the examples of this social proto-museum given by Cole is the Indianapolis Museum of Art (IMA), who established an ``Office of Art Grievances''. The office is there so that the public can file complaints about art, be it a singular piece or in general, which are processed through the bizarrely named ``Office of Art Resolution,'' in an attempt to create a ``feedback loop'' with the audience. The real problem with this, apart from the surface issue of a museum subjecting curators' informed decisions to the scrutiny of a largely unconcerned public, is the imminent danger of allowing museums to become ``safe spaces.'' Such transition would very clearly go against a large part of art, which aims to be transgressive and challenging to the public, rather than comfortable and unoffensive.

        Throughout the article Cole provides a set of examples of museums giving in to this social-experiment-esque approach, some of which with borderline hilarious proposals, such as Santa Cruz Museum's ``Everyone's Ocean.'' All of these together show us very clearly the author's point, leading us to comprehend just how meaningless the resulting exhibitions can be when curatorial work is ignored in favor of some ill-defined community will. Although all the evidence here is anecdotal, one can claim that it couldn't be otherwise since to analyze the success and qualities of exhibitions and museums one must, without exception, look at exhibitions and museums critically, and that therefore the anecdotal aspect of the argument is necessary for there to even be an argument at all, and not invalidating.
    
    %%%% Argumentation
        Cole's argument can be fundamentally described as making the intent of the post-modern approach to museums explicit, and then relying on how obviously ridiculous it is. By showing some of the proposals done by museums to do ``eccentric'' exhibition styles, such as the Indianapolis Museum of Art allowing the public to choose ``Hot Cars, High Fashion, Cool Stuff'' as a theme, it becomes clear just how silly and pointless the whole idea can be. The argument is done very clearly, mostly because the reader need not be convinced of the point, the evidence alone impels one to disagree with the techniques discussed. 

        Although Cole does not provide us with any alternative theories as to \emph{why} are museum directors forcing such a drastic functional shift, he seems to imply that they are simply following the trend, acting mindlessly and attempting to be ``cool'' and reinvent things. This, however, leaves his point open to the question of, were museums to maintain tradition, could they survive? I think a large number of museums have seen a steady decrease in visitations over the past decades, and it may just have reached a critical point where either they exist in such a way that attracts the general public, or not at all. And herein lies the true issue, which is that a large number of people simply aren't interested in ``higher art.'' This omission by Cole definitely leaves something to be desired from the article, and makes use wonder whether he is truly correct or not in assuming directors' obliviousness.

        One point that is softly made throughout the article is the critique to the grants given to these kinds of exhibitions. Cole mockingly describes the aim of the Walker's ``50/50: Audience and Experts Curate
        the Paper Collection,'' and immediately after doing so points out that they would be able to do more of the same with the help of a grant received from the Bush Foundation. The same sort of point is made with regard to the NEH directing funds to these post-museums, making it seem as if it were the tort ideas of its current Chairman. Although the point is crucial, for it concerns the taxpayer, it is not given as much attention as it should, which weakens the reality aspect of the problem as presented in the article.

        Regarding sample size, the article is adequate, but more examples of the problems given would have definitely helped. Throughout the piece 4 examples of town-hall museums are given, an amount which, although good, is still arguably prone to cherry-picking of data. One could easily claim that Cole selected 4 of the very worst cases of museum innovation and presented it as being the whole picture and, although that is not true, the article in and of itself does not have enough to defend against this. It would have been better had he refuted this idea in the article itself, or provided large-scale data that negated the possibility of cherry-picking.
    
    %%%% Background knowledge and received opinion
        As stated before, the article requires very little prior knowledge for one to comprehend and appreciate it. Bruce Cole's joyful writing style, with inserted jokes and irony throughout, make the read not only pleasant, but easy. Perhaps the only thing necessary to not just comprehend but to agree with the points made is to have been in a museum before, and to have a basic appreciation for art and the experiencing of it. The only point where he truly falls short is in considering \emph{why} is the shift truly happening, apart from the fact that directors want to follow trend. A simple analysis of the facts presented, namely that the IMA has a mini-golf course, can show very clearly that there is more to the issue than is presented, and this lack of information harms his point and could lead one to disagreeing with him entirely. 

    \end{mla}
    \end{document}