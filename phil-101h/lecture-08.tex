\documentclass{article}
    %%%%%%%%%%%%%%%%%%%%%%%%%%%%%%%%%%%%%%%%%%%%%%%%%%%%%%%%%%%%%
    % Lecture Specific Information to Fill Out
    %%%%%%%%%%%%%%%%%%%%%%%%%%%%%%%%%%%%%%%%%%%%%%%%%%%%%%%%%%%%%
    \newcommand{\LectureTitle}{Lecture \#8 Notes}
    %\newcommand{\LectureDate}{\today}
    \newcommand{\LectureDate}{February\ 26,\ 2017}
    \newcommand{\LectureClassName}{PHIL\ 101H}
    \newcommand{\LatexerName}{Bernardo\ Meurer}
    %%%%%%%%%%%%%%%%%%%%%%%%%%%%%%%%%%%%%%%%%%%%%%%%%%%%%%%%%%%%%

    % Change "article" to "report" to get rid of page number on title page
    \usepackage{amsmath,amsfonts,amsthm,amssymb}
    \usepackage{setspace}
    \usepackage{Tabbing}
    \usepackage{fancyhdr}
    \usepackage{lastpage}
    \usepackage{extramarks}
    \usepackage{chngpage}
    \usepackage{soul,color}
    \usepackage{graphicx,float,wrapfig}
    \usepackage{afterpage}
    \usepackage{abstract}
    \usepackage[margin=1in]{geometry}
    \usepackage{syllogism}


    % Setup the header and footer
    \pagestyle{fancy}
    \lhead{\LatexerName}
    \chead{\LectureClassName: \LectureTitle}
    \rhead{\LectureDate}
    \lfoot{\lastxmark}
    \cfoot{}
    \rfoot{Page\ \thepage\ of\ \pageref{LastPage}}
    \renewcommand\headrulewidth{0.4pt}
    \renewcommand\footrulewidth{0.4pt}

        %%%%%%%%%%%%%%%%%%%%%%%%%%%%%%%%%%%%%%%%%%%%%%%%%%%%%%%%%%%%%
    % Some tools
    \newcommand{\enterTopicHeader}[1]{\nobreak\extramarks{#1}{#1 continued on next page\ldots}\nobreak%
    \nobreak\extramarks{#1 (continued)}{#1 continued on next page\ldots}\nobreak}
\newcommand{\exitTopicHeader}[1]{\nobreak\extramarks{#1 (continued)}{#1 continued on next page\ldots}\nobreak%
   \nobreak\extramarks{#1}{}\nobreak}

\newlength{\labelLength}
\newcommand{\labelAnswer}[2]
{\settowidth{\labelLength}{#1}
\addtolength{\labelLength}{0.25in}
\changetext{}{-\labelLength}{}{}{}
\noindent\fbox{\begin{minipage}[c]{\columnwidth}#2\end{minipage}}
\marginpar{\fbox{#1}}

% We put the blank space above in order to make sure this
% \marginpar gets correctly placed.
\changetext{}{+\labelLength}{}{}{}}

\setcounter{secnumdepth}{0}
\newcommand{\TopicName}{}
\newcounter{TopicCounter}
\newenvironment{Topic}[1][Problem \arabic{TopicCounter}]
{\stepcounter{TopicCounter}
\renewcommand{\TopicName}{#1}
\section{\TopicName}
\enterTopicHeader{\TopicName}}
{\exitTopicHeader{\TopicName}}

\setcounter{secnumdepth}{0}
\newcommand{\ExampleSectionName}{}
\newcounter{ExampleSectionCounter}[TopicCounter]
\newenvironment{ExampleSection}[1][Example \arabic{ExampleSectionCounter}]
{\stepcounter{ExampleSectionCounter}
\renewcommand{\ExampleSectionName}{#1}
\section{\ExampleSectionName}
\enterTopicHeader{\ExampleSectionName}}
{\exitTopicHeader{\ExampleSectionName}}

\setcounter{secnumdepth}{0}
\newcounter{ExampleBoxCounter}[TopicCounter]
\newcommand{\examplebox}[1]
{
% We put this space here to make sure we're disconnected from the previous
% passage
\stepcounter{ExampleBoxCounter}
\noindent\fbox{\begin{minipage}[c]{\columnwidth}#1\end{minipage}}\enterTopicHeader{\ExampleSectionName}\exitTopicHeader{\ExampleSectionName}\marginpar{\fbox{\#\arabic{ExampleBoxCounter}}}
% We put the blank space above in order to make sure this
% \marginpar gets correctly placed.
\vskip10pt%
}

\renewcommand{\contentsname}{{\normalsize Topics Covered}}
\renewcommand{\abstractname}{\LectureTitle\ Summary}
\renewcommand{\absnamepos}{flushleft}

%%%%%%%%%%%%%%%%%%%%%%%%%%%%%%%%%%%%%%%%%%%%%%%%%%%%%%%%%%%%%
\begin{document}
\begin{spacing}{1.2}
    \newpage
    \section{Trolley Problems}
    \section{Sexism}
    \begin{itemize}
        \item Prejudice or discrimination based on sex or gender; especially discrimination against females and women.
        \item Behaviors, conditions, or attitudes that foster stereotypes of social roles based on sex or gender.
        \item Three forms of Sexism
              \begin{itemize}
                  \item Institutional sexism refers to individuous sexual inequalities in the explicit rules and implicit norms governing and structuring social institutions
                  \item Interpersonal sexism refers to interactions between persons that are not governed by explicit rules but create, constitute, promote, sustain, and/or exploit individuous sexual inequalities.
                  \item Unconscious sexism refers to the psychological mechanisms and tacit beliefs that that crate, constitute, promote, sustain and/or exploit individuous sexual inequalities.
              \end{itemize}
        \item Objections to Feminist struggles
              \begin{itemize}
                  \item Essentialism: There are biological and/or psychological differences between men and women such that true equality will be impossible (or too greatly constrain our liberty).
                  \item Skepticism: Sexism has been largely overcome.
                  \item Defeatism: There is nothing anyone can personally do to fight sexism; if you're not yourself mistreating women, then you're not required to do anything to combat sexism.
              \end{itemize}
    \end{itemize}
\end{spacing}
\end{document}