\documentclass[12pt,letterpaper]{article}
    \usepackage[utf8]{inputenc}
    \usepackage{ifpdf}
    \usepackage{mla}
    \usepackage[pass]{geometry}
    \usepackage{todonotes}
    \usepackage[hidelinks]{hyperref}

    \urlstyle{same}

\begin{document}
\begin{mla}{Bernardo}{Meurer}{Professor Eckford-Prossor}{English 111 H}{January 31, 2018}{Knowledge Chapter 2}
    \textbf{``\ldots an ordinary claim like `Here is a hand' is so basic that it is hard to find simpler and better-known claims we could use to support it. (There is a parallel with mathematics here, where some basic claims are taken as axioms, not themselves in need of proof)'' (Nagel 20).}
    \vspace{10pt}


    Here, while the author draws the mathematical analogy with the intent of supporting Moore's point (That the externality of one's hands is self-evident), the resulting effect is the exact opposite. As Gödel taught as in his incompleteness and inconsistency theorems, the axiomatic fundament of mathematics is not as sturdy as we naively expect it to be. Namely, Gödel shows that
    \begin{enumerate}
        \item If an axiomatic system is consistent, then it cannot be complete.
        \item The consistency of axioms cannot be proved within their own system.
    \end{enumerate}
    Therefore, just because an axiom doesn't require \emph{proof} that does not mean it is correct, which is to say the system in which it is embedded is consistent. As an example of this, the Axiom of Choice, formulated in 1904, still hasn't been incorporated into the Zermelo-Frankel axiomatic basis for set theory, in fear that it might make the system inconsistent.

    Looking back at the analogy then, because one can say ``Here is hand'', and it \emph{seems} self-evident and unprovable, that doesn't mean that it is \emph{correct}, for any reasonable definition of correctness. Moor here simply repeats the Descartian point of assuming a-priori that ``that which is show under the natural light is true'', or that which is self evident is true. Much like naïve set theory was proven grossly wrong, despite seeming perfectly right and reasonable, by Russel, this argument can probably be deconstructed in favor of the skeptical approach.

    Is it fair to assume that basic, ``axiomatic'', claims are self evident, and lack the burden of proof, for the sake of discussion, or is this a deforming simplification of the larger point?

    \begin{workscited}
        \bibent Nagel, Jennifer. Knowledge: a Very Short Introduction. Oxford University Press, 2014.
    \end{workscited}
\end{mla}
\end{document}