\documentclass[12pt,letterpaper]{article}
    \usepackage[utf8]{inputenc}
    \usepackage{ifpdf}
    \usepackage{mla}
    \usepackage[pass]{geometry}
    \usepackage{todonotes}
    \usepackage[hidelinks]{hyperref}

    \urlstyle{same}

\begin{document}
\begin{mla}{Bernardo}{Meurer}{Professor Eckford-Prossor}{English 111 H}{February 07, 2018}{Technopoly Chapter 9}
    ``\ldots that faith in science can serve as a comprehensive belief system that gives meaning to life, as well as a sense of well-being, morality, and even immortality.'' (Postman 147)
    \vspace{10pt}

    I don't see how this isn't true. Science does give meaning to life, insofar as we know it tells us it's meaningless, which in my opinion is a great thing, for it gives us the freedom to take meaning from life as we please, and not as dictated by some scripture. Science does not, however, give us a ``sense of well-being,'' but should it? Are discomfort, pain, and uncertainty not inalienable parts of the human condition? Why must we strive to avoid them, when often it is they who thrust us towards excellence?
    \begin{workscited}
        \bibent  Postman, Neil. \emph{Technopoly: the Surrender of Culture to Technology}. Vintage Books, 1993.
    \end{workscited}
\end{mla}
\end{document}