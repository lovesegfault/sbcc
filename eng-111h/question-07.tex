\documentclass[12pt,letterpaper]{article}
    \usepackage[utf8]{inputenc}
    \usepackage{ifpdf}
    \usepackage{mla}
    \usepackage[pass]{geometry}
    \usepackage{todonotes}
    \usepackage[hidelinks]{hyperref}

    \urlstyle{same}

\begin{document}
\begin{mla}{Bernardo}{Meurer}{Professor Eckford-Prossor}{English 111 H}{February 21, 2018}{Gulliver's Travels --- Book 1}
    We see the world through Ideological Lens, that cause us to observe the world around us as being ``normal'', in fact it is fair to say that Ideology is what causes us to define as normal that which we see daily and commonly. As an experiment to this, one can imagine the first Man who, in his explorations, came across a strange, short, tree with uncommonly large, glossy leafs, and from which hung, from some top branches that held a flower-like structure in its end, large groups of bow-shaped, cylinder-like, green and yellow semi-branches. Although the description above might seem like that of an extraterrestrial encounter with a nature far differing from our own, with some good-faith one can see that, in fact, the banana tree is what's being described. The commonness of the banana makes it seem ordinary, but most likely at first encounter and analysis it seems truly bizarre. To this filter and categorization of the normal, we call Ideology or Ideological lens (Zizek).
\end{mla}
\end{document}