\documentclass{article}
    %%%%%%%%%%%%%%%%%%%%%%%%%%%%%%%%%%%%%%%%%%%%%%%%%%%%%%%%%%%%%
    % Lecture Specific Information to Fill Out
    %%%%%%%%%%%%%%%%%%%%%%%%%%%%%%%%%%%%%%%%%%%%%%%%%%%%%%%%%%%%%
    \newcommand{\LectureTitle}{Lecture \#5 Notes}
    %\newcommand{\LectureDate}{\today}
    \newcommand{\LectureDate}{September\ 11,\ 2017}
    \newcommand{\LectureClassName}{PHIL\ 201}
    \newcommand{\LatexerName}{Bernardo\ Meurer}
    %%%%%%%%%%%%%%%%%%%%%%%%%%%%%%%%%%%%%%%%%%%%%%%%%%%%%%%%%%%%%
    
    % Change "article" to "report" to get rid of page number on title page
    
    \usepackage[utf8]{inputenc}
    \usepackage{amsmath,amsfonts,amsthm,amssymb}
    \usepackage{setspace}
    \usepackage{Tabbing}
    \usepackage{fancyhdr}
    \usepackage{lastpage}
    \usepackage{extramarks}
    \usepackage{chngpage}
    \usepackage{soul,color}
    \usepackage{graphicx,float,wrapfig}
    \usepackage{afterpage}
    \usepackage{abstract}
    
    % In case you need to adjust margins:
    \topmargin=-0.45in
    \evensidemargin=0in
    \oddsidemargin=0in
    \textwidth=6.5in
    \textheight=9.0in
    \headsep=0.25in
    
    % Setup the header and footer
    \pagestyle{fancy}
    \lhead{\LatexerName}
    \chead{\LectureClassName: \LectureTitle}
    \rhead{\LectureDate}
    \lfoot{\lastxmark}
    \cfoot{}
    \rfoot{Page\ \thepage\ of\ \pageref{LastPage}}
    \renewcommand\headrulewidth{0.4pt}
    \renewcommand\footrulewidth{0.4pt}
    
    %%%%%%%%%%%%%%%%%%%%%%%%%%%%%%%%%%%%%%%%%%%%%%%%%%%%%%%%%%%%%
    % Some tools
    \newcommand{\enterTopicHeader}[1]{\nobreak\extramarks{#1}{#1 continued on next page\ldots}\nobreak%
                                        \nobreak\extramarks{#1 (continued)}{#1 continued on next page\ldots}\nobreak}
    \newcommand{\exitTopicHeader}[1]{\nobreak\extramarks{#1 (continued)}{#1 continued on next page\ldots}\nobreak%
                                       \nobreak\extramarks{#1}{}\nobreak}
    
    \newlength{\labelLength}
    \newcommand{\labelAnswer}[2]
      {\settowidth{\labelLength}{#1}
       \addtolength{\labelLength}{0.25in}
       \changetext{}{-\labelLength}{}{}{}
       \noindent\fbox{\begin{minipage}[c]{\columnwidth}#2\end{minipage}}
       \marginpar{\fbox{#1}}
    
       % We put the blank space above in order to make sure this
       % \marginpar gets correctly placed.
       \changetext{}{+\labelLength}{}{}{}}
    
    \setcounter{secnumdepth}{0}
    \newcommand{\TopicName}{}
    \newcounter{TopicCounter}
    \newenvironment{Topic}[1][Problem \arabic{TopicCounter}]
      {\stepcounter{TopicCounter}
       \renewcommand{\TopicName}{#1}
       \section{\TopicName}
       \enterTopicHeader{\TopicName}}
      {\exitTopicHeader{\TopicName}}
    
    \setcounter{secnumdepth}{0}
    \newcommand{\ExampleSectionName}{}
    \newcounter{ExampleSectionCounter}[TopicCounter]
    \newenvironment{ExampleSection}[1][Example \arabic{ExampleSectionCounter}]
      {\stepcounter{ExampleSectionCounter}
       \renewcommand{\ExampleSectionName}{#1}
       \section{\ExampleSectionName}
       \enterTopicHeader{\ExampleSectionName}}
      {\exitTopicHeader{\ExampleSectionName}}
    
    \setcounter{secnumdepth}{0}
    \newcounter{ExampleBoxCounter}[TopicCounter]
    \newcommand{\examplebox}[1]
      {
      % We put this space here to make sure we're disconnected from the previous
       % passage
       \stepcounter{ExampleBoxCounter}
       \noindent\fbox{\begin{minipage}[c]{\columnwidth}#1\end{minipage}}\enterTopicHeader{\ExampleSectionName}\exitTopicHeader{\ExampleSectionName}\marginpar{\fbox{\#\arabic{ExampleBoxCounter}}}
       % We put the blank space above in order to make sure this
       % \marginpar gets correctly placed.
       \vskip10pt%
       }
    
    \renewcommand{\contentsname}{{\normalsize Topics Covered}}
    \renewcommand{\abstractname}{\LectureTitle\ Summary}
    \renewcommand{\absnamepos}{flushleft}
    
    %%%%%%%%%%%%%%%%%%%%%%%%%%%%%%%%%%%%%%%%%%%%%%%%%%%%%%%%%%%%%
    
    \begin{document}
    \begin{spacing}{1.2}
    \newpage
    \section{Meditation III}
    \textbf{What is required for him to be \underbar{certain} of anything?}

    Well, in the case of the \underbar{Cogito}, ``I saw very clearly that to 
    think one must exist.''

    ``What is required for knowledge is my simply having a clear and distinct 
    perception of what I am asserting.''
    
    Descartes provisionally then lays down the following as a general rule:
    \begin{itemize}
        \item ``Everything I very clearly and distinctly perceive is true.'', 
        in other words, ``Whatever the natural light shows me is true.''
        
        But yet, don't we clearly and distinctively perceive that \(2+2=4\)?
        Yet we cast doubt onto the validity of that statement in previous 
        meditations. Therefore, to assure myself that this rule may start 
        -- that it is reliable across-the-board -- I must be sure that God 
        exists (otherwise there could be an evil genius), and that he is not a 
        deceiver.

        Descartes proceeds to distinguish between judgments (can either be true or false) and ideas (cannot be true or false). Fundamentally, there are 
        3 types of ideas:
        \begin{itemize}
            \item Innate --- That I've always had
            \begin{itemize}
                \item Equality
                \item Existence
            \end{itemize}
            \item Adventitious --- That come to me unexpectedly
            \begin{itemize}
                \item Ideas that come unexpectly from external sources.
            \end{itemize}
            \item Invented --- Come from me
            \begin{itemize}
                \item Squaligator --- A fusion of a squirrel and an alligator
            \end{itemize}
        \end{itemize}
        He concludes that, is he is able to prove that God an innate idea, then 
        god must exist. 

        Take the idea of God. Eternal, infinite, immutable, omniscient, 
        omnipotent, etc. Isn't the idea of God objectively\footnotemark[1] more 
        real than any other idea I could possibly have. Even more than the idea 
        of myself. It represents the greatest amount of perfection. ``Is it not 
        evident by the light of nature that there must be as much reality in 
        the efficient [productive] and total cause as there is in the effect of 
        that same cause?'' In other words, there must be as much 
        formal\footnotemark[2] reality in the cause as there is objective 
        reality in the effect ...
        
    \end{itemize}
    \footnotetext[1]{Objective = representative}
    \footnotetext[2]{Formal = actual}
    \end{spacing}
    \end{document}