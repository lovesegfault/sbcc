%cSpell:words fancyhdr extramarks amsthm amsfonts algpseudocode usetikzlibrary hmwk Bobro Meurer topmargin evensidemargin oddsidemargin headsep linespread firstxmark lastxmark headrulewidth footrulewidth nobreak stepcounter setcounter secnumdepth newcounter Alph pagebreak megagon syllog

\documentclass{article}
    
    \usepackage[utf8]{inputenc}
    \usepackage{fancyhdr}
    \usepackage{extramarks}
    \usepackage{amsmath}
    \usepackage{amsthm}
    \usepackage{amsfonts}
    \usepackage{syllogism}
    
    %
    % Homework Details
    %   - Title
    %   - Due date
    %   - Class
    %   - Section/Time
    %   - Instructor
    %   - Author
    %
    
    \newcommand{\hmwkTitle}{Of Descartes' dependency on God}
    \newcommand{\hmwkDueDate}{September 25, 2017}
    \newcommand{\hmwkClass}{Philosophy 201}
    \newcommand{\hmwkClassTime}{}
    \newcommand{\hmwkClassInstructor}{Professor Marc Bobro}
    \newcommand{\hmwkAuthorName}{\textbf{Bernardo Meurer}}
    

    %
    % Basic Document Settings
    %
    
    \topmargin=-0.45in
    \evensidemargin=0in
    \oddsidemargin=0in
    \textwidth=6.5in
    \textheight=9.0in
    \headsep=0.25in
    
    \linespread{1.1}
    
    \pagestyle{fancy}
    \lhead{\hmwkAuthorName}
    \chead{\thepage}
    \rhead{\hmwkDueDate}
    \lfoot{}
    \cfoot{}
    
    \renewcommand\headrulewidth{0.4pt}
    \renewcommand\footrulewidth{0.4pt}
    
    \setlength\voffset{-0.15in}

    %
    % Title Page
    %
    
    \title{
        \vspace{2in}
        \textmd{{\hmwkClass}}\\
        \textbf{\hmwkTitle}\\
        \normalsize\vspace{0.1in}\small{Due\ on\ \hmwkDueDate\ at 3:55pm}\\
        \vspace{0.1in}\large{\textit{\hmwkClassInstructor\ \hmwkClassTime}}
        \vspace{3in}
    }
    
    \author{\hmwkAuthorName}
    \date{}
    
\begin{document}
\maketitle
\pagebreak
In Descartes' \textit{Meditations} he presents two distinct proofs for the 
existence of God. His first proof, being embedded in his third Meditation, is 
an attempt at solidifying some basis, however small, for knowledge. The second 
proof, an ode to Anselm's Ontological Argument, comes in his fifth Meditation 
and is the culmination of his efforts on the ``Theory of True and Immutable 
Natures,'' while also serving to provide his audience with a more scholastic, syllogistic proof. We shall make both proofs explicit, and show how his Philosophy is dependent on God, for the latter is the bedrock of the former.

It is crucial to comprehend Descartes' fundamental dependency on God, and to do 
so one can reference his efforts in \textit{Meditation III}, in which he envisions 
``Clear and distinct perception,'' things which are shown true to us by what he 
calls ``natural light,'' to be a valid basis for knowledge. He proposes that 
everything one clearly and distinctly perceives is true, and to validate this 
he must prove both that God exists, and that if so he is not a deceiver. He continues on with 
his proof of the existence God, which begins by recognizing one has an \emph
{idea} of God (eternal, infinite, omnipotent, etc). This idea, however, is 
objectively (representatively) more real than any other, it represents utter perfection,
 more real than even the idea of oneself. However, since we assumed that 
everything shown to use by \emph{natural light} is true, we can clearly see 
that it shall be true that there must be as much reality in a cause as as there 
is in the effect of said cause. This \emph{Causal Perfection Principle} 
fundamentally states that there must be as much formal reality in a cause as 
there is objective reality in its idea.

If we evaluate the idea of God once more we clearly see that it has more 
objective reality than ourselves, which would imply it must have come from 
something other than ourselves, showing God could not have been an Invented 
Idea. Following this thought it becomes inevitable that the concept of God must 
have come from something as \emph{perfect} as God himself, which can only be 
God. In summa, the fact that one has an idea of something entirely perfect, God,
means something just as perfect must in fact exist to cause it, which could 
only be God. Finally, Descartes argues that God cannot be a deceiver since we 
begin by assuming his perfection, and yet all fraud and deception must clearly 
come from a defect. This entangling of God and knowledge yields a clear and 
undeniable dependency of God for Descartes' Philosophy to be valid, for were 
there no God, then Clear and Distinct Perception could not be trusted.

In \textit{Meditation V} Descartes attempts to satisfy his audience by 
providing a more Aristotelian proof, and to do so he begins by showing how 
ideas have a \emph{true and immutable nature}. Consider the pentagon, 
independent of how free my imagination is all the pentagons I may idealize will 
share a set of common properties, they will all have 5 sides and the sum of 
internal angles will total \(540^{\text{o}}\). All pentagons have an essence, a 
nature, which is independent of me and is immutable. Moreover the existence of 
said nature is entirely independent of my ability to visualize the entity in 
question. The Megagon, a regular polygon with a million sides, is far beyond 
our imagination, and yet we can make fundamentally true statements about its 
nature, such as saying that each of its internal angles measures \(179.99964^
{\text{o}}\). One can very clearly and distinctly perceive these properties, 
and therefore one cannot doubt that they do, in fact, belong to it. It is also 
clear that said properties are not invented by ourselves, they require our 
thought and analysis to come to light, and one can never be sure whether all 
properties have been found or not.

Finally, Descartes applies his concept of true and immutable nature to God 
itself, meaning that if one clearly and distinctly perceives a property of God, 
then said property \emph{must} belong to God. He then constructs his syllogism:
\begin{center}
    \syllog{God is utterly perfect}%
    {A lack of existence is a lack of perfection}%
    {God does not lack existence} 
\end{center}
Fundamentally we conclude that one can be just as certain of the existence of God as one can be of the sum of angles of the pentagon, or of the internal angle of the regular Megagon. On this line, it follows that if there are polygons, and we can determine their properties and show them as being immutable, then there is also God and he exists.

\end{document}  