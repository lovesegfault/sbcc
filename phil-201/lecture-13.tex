\documentclass{article}
    %%%%%%%%%%%%%%%%%%%%%%%%%%%%%%%%%%%%%%%%%%%%%%%%%%%%%%%%%%%%%
    % Lecture Specific Information to Fill Out
    %%%%%%%%%%%%%%%%%%%%%%%%%%%%%%%%%%%%%%%%%%%%%%%%%%%%%%%%%%%%%
    \newcommand{\LectureTitle}{Lecture \#13 Notes}
    %\newcommand{\LectureDate}{\today}
    \newcommand{\LectureDate}{October\ 23,\ 2017}
    \newcommand{\LectureClassName}{PHIL\ 201}
    \newcommand{\LatexerName}{Bernardo\ Meurer}
    %%%%%%%%%%%%%%%%%%%%%%%%%%%%%%%%%%%%%%%%%%%%%%%%%%%%%%%%%%%%%

    % Change "article" to "report" to get rid of page number on title page

    \usepackage[utf8]{inputenc}
    \usepackage{amsmath,amsfonts,amsthm,amssymb}
    \usepackage{setspace}
    \usepackage{Tabbing}
    \usepackage{fancyhdr}
    \usepackage{lastpage}
    \usepackage{extramarks}
    \usepackage{chngpage}
    \usepackage{soul,color}
    \usepackage{graphicx,float,wrapfig}
    \usepackage{afterpage}
    \usepackage{abstract}

    % In case you need to adjust margins:
    \topmargin=-0.45in
    \evensidemargin=0in
    \oddsidemargin=0in
    \textwidth=6.5in
    \textheight=9.0in
    \headsep=0.25in

    % Setup the header and footer
    \pagestyle{fancy}
    \lhead{\LatexerName}
    \chead{\LectureClassName: \LectureTitle}
    \rhead{\LectureDate}
    \lfoot{\lastxmark}
    \cfoot{}
    \rfoot{Page\ \thepage\ of\ \pageref{LastPage}}
    \renewcommand\headrulewidth{0.4pt}
    \renewcommand\footrulewidth{0.4pt}

    %%%%%%%%%%%%%%%%%%%%%%%%%%%%%%%%%%%%%%%%%%%%%%%%%%%%%%%%%%%%%

    \begin{document}
    \begin{spacing}{1.2}
    \newpage
        \section*{Gottfried Leibniz}
        \begin{flushright}
            1646 -- 1716
        \end{flushright}
        \subsection*{Primary Truths}
        All statements involve a subject and predicate.
        Take

        ``Washington crossed the Delaware''

        \noindent Here ``Washington'' is the subject, and ``crossed the Delaware'' is the predicate.
        A true statement is one where the predicate \emph{belongs} to the subject. In every true statement, the concept of the predicate is contained in the concept of the subject. This is called Leibiz's Concept-Containment Theory of Truth.
        Let's take some trivial examples:

            ``A is A''

        \noindent Take the concept of A, and it belongs to that concept is being A

            ``Everything is what it is''

        \noindent These are obvious examples, and that's because these are statements of identity, meaning that the subject-concept and the predicate-concept are identical. The predicate is obviously contained in these examples. Such truths are called \emph{primary truths}. Secondary proofs, in contrast, are ones where the predicate is \emph{not} obviously contained. All truths can be reduced to primary truths through an analysis of the subject-concept. All truths are analytic, whether universal (``All dogs are mammals''), particular (``Lassie is not a cat''), necessary (``God exists''), or contingent (``Washington crossed the Delaware'').

        ``In a necessary truth the predicate-concept can be seen to be contained in the subject-concept with a finite analysis''

        Some necessary truths: \(2+2=4\), squares have fours sides, A is A, God Exists
        Some contingent truths: Washington crossed the Delaware, Spinoza died in the Hague, I exist.

        In a contingent truth, the predicate-concept can be seen to be contained in the subject-concept only by using an infinite analysis.

        Important conclusions:
        \begin{enumerate}
          \item Nothing is without reason. There is no effect without a cause. (Principle of Sufficient Reason)
          \begin{itemize}
            \item Buridan's Ass
          \end{itemize}
          \item In nature, there cannot be two individual things that differ in number alone. Two numerically distinct things cannot resemble each other completely. (Principle of The Identity of Indiscernibles)
          \begin{itemize}
              \item ``There has to be a qualitative difference for there to be a quantitative difference.''
          \end{itemize}
          \item There are no purely extrinsic denominations.
          \begin{itemize}
              \item The subject concept has to be complete.
          \end{itemize}
          \item Every substance contains in its complete concept the entire universe and everything that exists in it, past, present, and future.

          \item All substances that exist (that are created) are expressions of the same universe.

          $$\mathcal U = \{\mathcal C \colon \epsilon\in\mathcal C\}$$
          Where $\mathcal U$ is the universe, and $\epsilon$ is the predicate of existence.

          \item No created substance causally affects any other substance. Our minds or bodies, and conversely.

          \item There is a pre-established harmony between mind and body
        \end{enumerate}

        Idea: Empty space/vacuum correlated with the proof that every set contains the empty set.

        Concept-containment gives rise to the problem of freedom. Since things are predtermined by a concept that precedes even birth, then how is there any free-will? How can we account for human freedom?


        (According to Leibiz some truths (2+2=4) will always be true. )
    \end{spacing}
    \end{document}