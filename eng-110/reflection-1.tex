\documentclass[12pt,letterpaper]{article}
\usepackage[utf8]{inputenc}
\usepackage{ifpdf}
\usepackage{mla}
\usepackage{todonotes}
\usepackage{setspace}

\begin{document}
\begin{mla}{Bernardo}{Meurer}{Professor Martin}{English 110}{October 24, 2017}{Reflection}
\begin{singlespace}
    \textbf{``My name is Lillian, and thank you for being so honest. When your friend, or whatever he is, arrives, I'll turn off my hearing aids so you'll have privacy''}
\end{singlespace}

This quote elucidates a characteristic of the story that I find very odd when reading, namely its seeming lack of worry about verisimilitude in the human interactions. Most of the dialogues in the text have this extremely odd and unnatural feel to them, they don't sound or read like real people talking, and it breaks the immersion into the story. Another example of this point lies in Corliss' exchange with the homeless man. Although the man makes a very interesting and eloquent point about his own humanity (or lack thereof), it isn't introduced fluidly into the dialogue at all, it reads in such a way that they both seem almost robotic. Although one could claim that this is intentional, it doesn't seem that way since Corliss is depicted throughout the story as a lively, opinionated young woman, which contradicts the almost industrial aspect of her dialogues. Here, when I say industrial I mean it like pasteurized, it feels as if some of the conversations had been written by a story machine that guarantees maximum consistency at the cost of any excellence, they just don't feel involved or organic, but rather are very clearly fabricated. It is also interesting to note that this ``coldness'' does not translate into her monologues, those are indeed very fitting of her character as molded by the story. This leaves me begging the question of whether the non-naturalistic and awkward dialogues carry some deeper meaning into them, or if they are simply an acute case of unfortunate writing?

\end{mla}
\end{document}