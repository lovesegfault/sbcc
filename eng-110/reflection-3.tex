\documentclass[12pt,letterpaper]{article}
    \usepackage[utf8]{inputenc}
    \usepackage{ifpdf}
    \usepackage{mla}
    \usepackage{todonotes}
    \usepackage{setspace}
    
    \begin{document}
    \begin{mla}{Bernardo}{Meurer}{Professor Martin}{English 110}{November 9, 2017}{Reflection}
    \begin{singlespace}
        \textbf{``So I guess you were hopelessly romantic and easily distracted, a B-plus mother, certainly good enough to get into Matriarchal State University but not quite good enough for St. Mary’s College of the Blessed Womb Warriors.''}
    \end{singlespace}
    
    Here we see, at this point borderline miraculously, a joke that does not induce the reader a strong will to close the book and never look back. It is interesting to see that Alexie can, in fact, be funny, and it leaves the question of why did he not use this more throughout the book? Sure, there are jokes in the previous stories, but they all suffer from being just painfully unfunny and nauseating, they're all inorganic and at times repulsive. Even in this story, when the subject describes his mothers' insults, Alexie uses his usual failing humor. It is uncertain to me, given that the author can clearly write a good joke, how the book manages to be so bad, specially in stories such as ``Do Not Go Gentle,'' where the humor is so forced and tries so hard that it might give some a flashback of their school years, where a ``class clown'' made people laugh by pulling increasingly stupid (and ultimately sad) stunts. That clown eventually grows out of it, repents on his behavior, and generally moves on with life, hopefully this is what's happening to Alexie at this point in the book, acceptance and improvement.
    
    \end{mla}
    \end{document}