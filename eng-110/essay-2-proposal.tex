\documentclass[12pt,letterpaper]{article}
    \usepackage[utf8]{inputenc}
     \usepackage{ifpdf}
    \usepackage{mla}
    \usepackage{todonotes}
    
    
        \begin{document}
        \begin{mla}{Bernardo}{Meurer}{Professor Martin}{English}{October 3, 2017}%
            {Essay \#2 Proposal}
        In the Digital Era, Copyright has become harder to enforce than ever. 
        Digital works are subject to copying without any quality degradation, which encourages piracy and can lead to market failure.  DRM (Digital Rights Management) is, according to some, a successful implementation of copyright in the digital world, which protects both markets and creators alike. The truth, however, is quite the opposite, since DRM unjustly and broadly hardens access to copyrighted works. As Favale argues in ``Death and resurrection of copyright between law and technology'', ``as long as digital locks can be implemented in a way to respect the entitlements of the users, they must do so or they should not be implemented at all.'' This is exactly the point to be made, that DRM should either exist in a moral, controlled way, or not at all.
    
        The cause for today's dreadful state of digital copyright is mostly due to a lack of attention to fundamental property rights (in both the Lockean and Hegelian sense) on the part of millennial legislation on the matter (DMCA, et al.)  Catherine Lemmer \& Carla Lanham put this eloquently in their work ``Digital Rights Management: The Librarian's Guide'', where they state ``DRM has become less about copyright and more about protecting owners' profits''. This, in fact, ties in well with the Lockean-Hegelian analysis of intellectual property that Wendy J. Gordon elucidates in her essay ``A Property Right in Self-Expression: Equality and Individualism in the Natural Law of Intellectual Property'', where she argues ``People must have access not only to the physical common of which Locke explicitly writes, but also to a common of intangibles. Our common encompasses not only our physical country but our culture as well.''

        The over-bearing implementation of Digital Copyright, in the form of DRM, is intrinsically wrong for it contradicts the fundamental reason for the existence of copyright itself. The Copyright Act 1710, also known as the Statute of Anne, was the first attempt at a copyright legislation. It's title sheds light on the true intention of copyright,
        it reads ``An Act for the Encouragement of Learning, ...'' And yet, as Ann Ludbrook points out in ``Canada's E-book Withdrawal: Digital Rights Management and the Canadian Electronic Library'', DRM has quite the opposite effect than encouraging learning, as she says ``DRMs, such as view-only display, can affect the ability of even short excerpts of an e-book to be used under fair dealing for education.'' Moreover, as Favale points out ``current copyright law resources to several contrivances to trump public entitlements and to skew copyright protection in favour of rightholders.''

        I need to better comprehend at which point the interest of the community overcomes the interest of the individual in Locke's theory. Also, further investigation of Hegel's Theory of Right is needed, to deepen my understanding of how property can be seen as the expression of Man's personality. Also, more examples of harmful DRM implementations would be very useful.
        \end{mla}
        \end{document}