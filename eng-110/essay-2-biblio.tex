\documentclass[12pt,letterpaper]{article}
    \usepackage[utf8]{inputenc}
     \usepackage{ifpdf}
     \usepackage{hyperref}
    \usepackage{mla}
    \usepackage{todonotes}
    
    
        \begin{document}
        \begin{mla}{Bernardo}{Meurer}{Professor Martin}{English}{October 3, 2017}%
            {Annotated Bibliography}
            \bibent
            Favale, Marcella. ``Death and Resurrection of Copyright between Law and Technology.'' Information \& Communications Technology Law, vol. 23, no. 2, June 2014, pp. 117-135. EBSCOhost, doi:10.1080/13600834.2014.925631.

            Favale writes a truly amazing overview of digital property rights and DRM. She makes very fundamental points, bringing Locke's arguments to light, and cites original legislation on the subject. This source is instrumental for my argumentation.

            \bibent 
            Deveci, Hasan A. ``Can Hyperlinks and Digital Rights Management Secure Affordable Access to Information?.'' Computer Law \& Security Review, vol. 28, no. 6, Dec. 2012, pp. 651-661. EBSCOhost, doi:10.1016/j.clsr.2012.09.002.

            Deveci makes some interesting points about the shifting of copyright's effects over the years, from nourishing development to stemming it. This work can be used as an auxiliary to the point that current implementations of DRM alienate the user.

            \bibent
            Zwollo, Kim. ``Digital Reuse Rights in a Fast-Changing Publishing Environment: Content Users' Demands and Licensing Options for Publishers.'' Information Services \& Use, vol. 32, no. 1/2, Jan. 2012, pp. 79-83.

            Zwollo makes strong points correlating DRM and it's malfunctioning and/or difficulty of use with copyright infringement. He argues for a simplified digital copyright model that strives for sanity. This is really useful both in bringing in statistics to complement the philosophical arguments as well as in defending the idea that DRM does not work.

            \bibent
            Felten, Edward W. ``DRM and Public Policy.'' Communications of the ACM, vol. 48, no. 7, July 2005, p. 112. EBSCOhost, doi:10.1145/1070838.1070871.

            Felten fundamentally answers the question of ``How to make DRM work?'', and does so in an extremely concise form. He outlines 6 principles for sensible public policies regarding DRM, which will be extremely useful for the conclusion.

            \bibent
            Gordon, Wendy J. ``A Property Right in Self-Expression: Equality and Individualism in the Natural Law of Intellectual Property.'' The Yale Law Journal, vol. 102, no. 7, 1993, pp. 1533–1609. JSTOR, JSTOR, www.jstor.org/stable/796826.

            Gordon goes very deep into the Lockean and Hegelian analysis for intellectual property rights, and brings them into the digital age. Her work is instrumental, in conjunction to Favale's article.

            \bibent
            Lemmer, Catherine A, and Carla P Wale, editors. ``Digital Rights Management : The Librarian's Guide''. Lanham, Rowman \& Littlefield, 2016.

            Lemmer and Wale make the conflicts between DRM and public interest very clear in their writings, and they are not afraid to point out openly the bigotry of DRM. Their points are useful in arguing that DRM serves only the Capital, and not the people nor the creator.

            \bibent
            Locke, John. ``Two Treatises of Government and A Letter Concerning Toleration'', edited by Ian Shapiro, Yale University Press, 2003. ProQuest Ebook Central, https://ebookcentral.proquest.com/lib/sbcc-ebooks/detail.action?docID=3420119.

            In his famous work Locke makes a plethora of arguments regarding property and individual rights, all of which can be used in one way or the other. I plan on bringing forth his arguments that give entitlements to the user.

            \bibent
            Hegel, Georg Wilhelm Friedrich and Stephen Houlgate. ``Outlines of the Philosophy of Right''. OUP Oxford, 2008. Oxford World's Classics.

            Hegel makes interesting points that will be the foundation for \emph{droit d'auter}. In particular his argument that property is the expression of Man's personality is very interesting. His work will be useful in showing that intellectual property is not different from material property insofar as rights are concerned.

        \end{mla}
        \end{document}