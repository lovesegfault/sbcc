\documentclass[12pt,letterpaper]{article}
    \usepackage[utf8]{inputenc}
    \usepackage{ifpdf}
    \usepackage{mla}
    \usepackage{todonotes}
    \usepackage[hidelinks]{hyperref}

    \urlstyle{same}

    \begin{document}
    \begin{center}
        \vfill
        \huge{Note}\\
        \normalsize
        None of the thoughts or opinions presented here are necessarily my own. The position taken in this essay was picked to maximize potential grade, and not to be a truthful representation of my opinions on the book, the author, or anything else.
        \begin{flushright}
            --- Bernardo Meurer
        \end{flushright}
        \vfill
    \end{center}
    \newpage
    \setcounter{page}{1}
    \begin{mla}{Bernardo}{Meurer}{Professor Martin}{English 110}{November 27, 2017}%
        {On the Ceremonialism of the Native American}
        %%%% Intro
        In Sherman Alexie's unquestionable masterpiece ``Ten Little Indians,'' we are presented with nine fictitious stories of Native Americans, in conjunction with the author's autobiography (thus completing the ten Indians.) Throughout the work, we are faced with beautiful, engaging stories of modern Native Americans struggling to live their lives in a world created by, and for ``White People,'' who usurped the Native Americans from their rightful land and birthplace, having contained them to a few reservations across the country. American Indians are an extremely damaged community because of how brutal and unjust their interaction with the White Man was, and to this day, as is shown in the stories, they have not recovered from the damage that was done. Although all the stories are disparate, having different timelines and protagonists, there are a few factors that sow them together. One such factor, however, sows the characters with the entire Native American id (in the Freudian sense): ceremonies.

        A prime example of the use of ceremonies is in the story ``Do Not Go Gentle,'' a true work of art and subtlety which presents to us the story of parents in distress over their dying child. It is of no question that the death of a child, or on the least severe sickness, is of tremendous impact in one's life, and it is exactly in moments like this, of inconstancy and uncertainty, that one needs something to ground oneself in reality, something constant and familiar. Often times we will see this craving for the common in times of distress materialize itself in things such as prayer, or intense dedication to work; in ``Do Not Go Gentle'', however, it shows itself through a modification of the ancient ritual of the powwow. In a moment of breakdown, the father of the diseased child uses a dildo, which he had purchased after accidentally entering a sex shop thinking it was a toy store, to perform a powwow dance on the children's hospital. As he danced with the great brown dildo and other parents joined in, the magic of the ancient ceremony filled the halls and ``Every sick and dying and alive and dead kid heard it, and they were happy and good in their hearts. My wife sang the most beautiful song anybody ever heard in that place. She sang like ten thousand Indian grandmothers rolled into one mother.'' Here we can see very clearly the sense of empowerment given to the parents by the ceremony, it is a tool for them to exert control over their children's fate, something that is utterly out of their hands.

        Another interesting story under the ritualistic optic is ``What You Pawn I Will Redeem,'' the epic saga of Jackson Jackson to recover his grandmother's powwow regalia, which he found displayed in a pawn shop for sale. Here once again we see a setting that follows the same theme as ``Do not Go Gentle'': misery. This is made clear to us at the very beginning of the story where we read ``I didn’t break hearts into pieces overnight. I broke them slowly and carefully. I didn’t set any land-speed records running out the door. Piece by piece, I disappeared. And I’ve been disappearing ever since \ldots I’ve been homeless for six years.'' Once again we have the thematic of a life in some form of distress or chaos, and in which one must grasp onto something to be grounded in reality. Throughout the story Jackson mentions how he, and his life, began to deteriorate after his grandmother, Agnes, died of breast cancer when he was a teenager. Very clearly, therefore, his quest to reclaim her regalia is, in fact, a much deeper quest to reclaim her, and himself through connecting to his own people via that regalia and the rituals associated with it. This culminates on the end of the story where, reunited with the regalia, Jackson performs a ceremonial dance, ``I took my grandmother’s regalia and walked outside. I knew that solitary yellow bead was part of me. I knew I was that yellow bead in part. Outside, I wrapped myself in my grandmother’s regalia and breathed her in. I stepped off the sidewalk and into the intersection. Pedestrians stopped. Cars stopped. The city stopped. They all watched me dance with my grandmother. I was my grandmother, dancing.'' Once again we clearly see that through ritual, dancing with the regalia, one can reconnect with deep and forgotten memories. Ceremonies empower us with the energy of all the thousands of people who performed us before us.

        Finally, we must look at ``Do You Know Where I Am?'' the story of a loving Indian couple living and succeeding in the White man's world. Here, although at first it might be tempting to state that there is no real situation of distress or chaos, if we look underneath the surface, said distress becomes apparent. The life of a Native American in the bigoted world of the White Man is itself the distress. Even without some precarious condition, to simply live under the oppression of a culture that is not your own is traumatizing, and calls for rituals to ground one to one's roots. In this story this ritual happens on their routine with their children, namely, ``Sharon and I danced with our children. We danced the family dance, three quick spins, two hops, and a scream at the ceiling, and then Sharon and I made dinner, and we ate with our kids and gossiped about their school days and played Chutes and Ladders and watched The Lion King \ldots'' Very clearly here we have embedded among more typical practices (Playing Chutes and Ladders and watching The Lion King), an embedding of traditional Native American culture (having a family dance), which serves the ultimate purpose of connecting them to their roots and grounding them to the reality of who they are. Their kids might have been born in a city, and they might go to a school where they are the only Indians, and they might never have stepped on a reservation, but despite any of that they are true Indians for they share the ceremonies that bind them with their ancestors and community, it defines them and it links them to their identity.

        In conclusion, the characters in Sherman Alexie's masterpiece are all bound together by their use of ceremony as a method for reconnecting with their respective communities and identities. Whether it is by dancing, performing powwow, or any number of other possible rituals, they all perform the same fundamental task of joining one with the thousands of other people who have, throughout the ages, performed that ceremony and created a common identity that they can all tap into. A ceremony is a magical act of speaking, through actions, to people whom you've never met, but nonetheless form your concept of yourself through ethnic and group identity. Alexie's characters, like many other Native Americans in distraught from the irreparable damage caused by the White Man during the genocide of their people, use ceremonies to embrace their identity, and connect to the thousands of man and woman of their kinship who were brutally murdered, by lead or by alcohol, and whom they long deeply for.
        \begin{flushright}
            \textbf{1202 words}
        \end{flushright}
        \begin{workscited}
            \bibent
            Alexi, Sherman. ``Ten Little Indians.'' New York: Grove Press. 2003.
        \end{workscited}
    \end{mla}
    \end{document}