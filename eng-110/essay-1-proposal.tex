\documentclass{article}
    
    \usepackage[utf8]{inputenc}
    \usepackage{fancyhdr}
    \usepackage{extramarks}
    \usepackage{amsmath}
    \usepackage{amsthm}
    \usepackage{amsfonts}
    \usepackage{enumitem}
    \usepackage{tikz}
    \usepackage[plain]{algorithm}
    \usepackage{algpseudocode}
    
    \usetikzlibrary{automata,positioning}
    
    %
    % Basic Document Settings
    %
    
    \topmargin=-0.45in
    \evensidemargin=0in
    \oddsidemargin=0in
    \textwidth=6.5in
    \textheight=9.0in
    \headsep=0.25in
    
    \linespread{1.1}
    
    \pagestyle{fancy}
    \lhead{\hmwkAuthorName}
    \chead{\hmwkClass\ (\hmwkClassInstructor): \hmwkTitle}
    \rhead{\firstxmark}
    \lfoot{\lastxmark}
    \cfoot{\thepage}
    
    \renewcommand\headrulewidth{0.4pt}
    \renewcommand\footrulewidth{0.4pt}
    
    \setlength\parindent{0pt}
    
    %
    % Create Problem Sections
    %
    
    \newcommand{\enterProblemHeader}[1]{
        \nobreak\extramarks{}{Problem \arabic{#1} continued on next page\ldots}\nobreak{}
        \nobreak\extramarks{Problem \arabic{#1} (continued)}{Problem \arabic{#1} continued on next page\ldots}\nobreak{}
    }
    
    \newcommand{\exitProblemHeader}[1]{
        \nobreak\extramarks{Problem \arabic{#1} (continued)}{Problem \arabic{#1} continued on next page\ldots}\nobreak{}
        \stepcounter{#1}
        \nobreak\extramarks{Problem \arabic{#1}}{}\nobreak{}
    }
    
    \setcounter{secnumdepth}{0}
    \newcounter{partCounter}
    \newcounter{homeworkProblemCounter}
    \setcounter{homeworkProblemCounter}{1}
    \nobreak\extramarks{Problem \arabic{homeworkProblemCounter}}{}\nobreak{}
    
    %
    % Homework Problem Environment
    %
    % This environment takes an optional argument. When given, it will adjust the
    % problem counter. This is useful for when the problems given for your
    % assignment aren't sequential. See the last 3 problems of this template for an
    % example.
    %
    \newenvironment{homeworkProblem}[1][-1]{
        \ifnum#1>0
            \setcounter{homeworkProblemCounter}{#1}
        \fi
        \section{Problem \arabic{homeworkProblemCounter}}
        \setcounter{partCounter}{1}
        \enterProblemHeader{homeworkProblemCounter}
    }{
        \exitProblemHeader{homeworkProblemCounter}
    }
    
    %
    % Homework Details
    %   - Title
    %   - Due date
    %   - Class
    %   - Section/Time
    %   - Instructor
    %   - Author
    %
    
    \newcommand{\hmwkTitle}{Essay 1 Proposal}
    \newcommand{\hmwkDueDate}{September 6, 2017}
    \newcommand{\hmwkClass}{English}
    \newcommand{\hmwkClassTime}{}
    \newcommand{\hmwkClassInstructor}{Professor Joanne Martin}
    \newcommand{\hmwkAuthorName}{\textbf{Bernardo Meurer}}
    
    %
    % Title Page
    %
    
    \title{
        \vspace{2in}
        \textmd{\textbf{\hmwkClass:\ \hmwkTitle}}\\
        \normalsize\vspace{0.1in}\small{Due\ on\ \hmwkDueDate\ at 3:55pm}\\
        \vspace{0.1in}\large{\textit{\hmwkClassInstructor\ \hmwkClassTime}}
        \vspace{3in}
    }
    
    \author{\hmwkAuthorName}
    \date{}
    
    \renewcommand{\part}[1]{\textbf{\large Part \Alph{partCounter}}\stepcounter{partCounter}\\}
    
\begin{document}
\maketitle
\pagebreak
\begin{homeworkProblem}
    \textbf{Who is your audience? What kind of tone/language should you use in 
    this essay?}

    My audience is the academic community, and the essay should be written in 
    formal language.
\end{homeworkProblem}
\begin{homeworkProblem}
    \textbf{What does your audience need to know (introduction) before you 
    state your thesis?}

    They need to know \emph{what} the issue with College is, they need to 
    placed into the context of the issue. 
\end{homeworkProblem}
\begin{homeworkProblem}
    \textbf{What is your working thesis?}
    
    Although College is expensive, it is well worth it's high price.
\end{homeworkProblem}
\begin{homeworkProblem}
    \textbf{What specific points will you need to argue in order to support 
    your thesis?}

    \begin{enumerate}
        \item College is a worthy investment, it yields long-term monetary gains
        \item College has intangible benefits that make you a better citizen 
        and person
        \item College protects you from unemployment
        \item College provides you with a powerful network.
    \end{enumerate}
\end{homeworkProblem}
\begin{homeworkProblem}
    \textbf{Which articles will you use to support your argument?}
    \begin{itemize}
        \item ``What's the value of a college education'' by Jennifer Barrett
        \item ``Is college worth it? Clearly, new data say'' by David Leonhart
        \item ``Is it still worth going to college?'' by Mary C. Daly \& Leila 
        Bengali 
    \end{itemize}
\end{homeworkProblem}
\begin{homeworkProblem}
    \textbf{Provide at least one specific example from each article that you 
    might use in your essay. Explain how each of these examples will support 
    your ideas.}

    \begin{enumerate}
        \item ``What's the value of a college education'' by Jennifer Barrett
        \begin{itemize}
            \item ``Almost all of the highest-paying majors are in engineering 
            fields.'' --- Engineering majors require going to college, this 
            supports point 1.\ described in Problem 4. 
            \item ``the country will need nearly 250,000 more engineers over the next 10 years to work in high-growth sectors and industries.'' --- This falls in line with point 3.\ clearly.
        \end{itemize}
        \newpage
        \item ``Is college worth it? Clearly, new data say'' by David Leonhart
        \begin{itemize}
            \item ``The pay of people with a four-year college degree has risen 
            compared to that of those with a high school degree but no 
            college.'' --- Speaks to point 1.
            \item ``The pay gap between college graduates and everyone else reached a record high last year [2013]'' --- Also point 1.
        \end{itemize}
        \item ``Is it still worth going to college?'' by Mary C. Daly \& Leila 
        Bengali
        \begin{itemize}
            \item ``For most Americans the path to higher future earnings 
            involves a four-year college degree.'' --- Points 1.\ and 3.
            \item ``The average college graduate can recover the costs of 
            attending in less than 20 years.'' --- Good point in general, relates well to thesis statement.
        \end{itemize}
    \end{enumerate}
\end{homeworkProblem}
\begin{homeworkProblem}
    \textbf{What specific personal examples can you provide to help argue this idea?}

    Personal examples generally constitute anecdotal evidence, they are prone to cherry-picking and false-analogy fallacies, and are fundamentally weaker than real data. I don't intend on using it on my 
    arguments.
\end{homeworkProblem}
\begin{homeworkProblem}
    \begin{enumerate}[label=\alph*)]
        \item \textbf{What are some possible opposing argument(s) to your position?}
        \begin{enumerate}
            \item The price of college is too high
            \item College takes too long
            \item College has no bearing on your success
        \end{enumerate}
        \item \textbf{What articles will you use to demonstrate the opposing argument(s)?}
        \begin{enumerate}
            \item  ``We send too many students to college'' by Marty Nemko
            \item ``College is a waste of time'' by Dale Stephens
            \item ``Practical experience trumps fancy degrees'' by Tony Brummel
        \end{enumerate}

        \item \textbf{How will you refute those arguments?}
        \begin{enumerate}
            \item College may be expensive, but the cost is well worth it, and 
            you will get a return on your investment, even if perhaps not on 
            the short-term.
            \item The U.S. median life expectancy is of 80 years. The 4 years of college are a mere 5\% of that, and it has a highly positive impact on the future years.
            \item The argument is straight out fallacious, it's cherry picked and anecdotal. Statistically college degrees \emph{do} imply a better, more successful life. 
        \end{enumerate}

        \item \textbf{Will you use any articles to refute those opposing arguments? Which ones?}
        \begin{enumerate}
            \item Probably not.
        \end{enumerate}

    \end{enumerate}
\end{homeworkProblem}

\end{document}  