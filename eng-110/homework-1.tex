\documentclass{article}
    
    \usepackage[utf8]{inputenc}
    \usepackage{fancyhdr}
    \usepackage{extramarks}
    \usepackage{amsmath}
    \usepackage{amsthm}
    \usepackage{amsfonts}
    \usepackage{tikz}
    \usepackage[plain]{algorithm}
    \usepackage{algpseudocode}
    
    \usetikzlibrary{automata,positioning}
    
    %
    % Basic Document Settings
    %
    
    \topmargin=-0.45in
    \evensidemargin=0in
    \oddsidemargin=0in
    \textwidth=6.5in
    \textheight=9.0in
    \headsep=0.25in
    
    \linespread{1.1}
    
    \pagestyle{fancy}
    \lhead{\hmwkAuthorName}
    \chead{\hmwkClass\ (\hmwkClassInstructor): \hmwkTitle}
    \rhead{\firstxmark}
    \lfoot{\lastxmark}
    \cfoot{\thepage}
    
    \renewcommand\headrulewidth{0.4pt}
    \renewcommand\footrulewidth{0.4pt}
    
    \setlength\parindent{0pt}
    
    %
    % Create Problem Sections
    %
    
    \newcommand{\enterProblemHeader}[1]{
        \nobreak\extramarks{}{Problem \arabic{#1} continued on next page\ldots}\nobreak{}
        \nobreak\extramarks{Problem \arabic{#1} (continued)}{Problem \arabic{#1} continued on next page\ldots}\nobreak{}
    }
    
    \newcommand{\exitProblemHeader}[1]{
        \nobreak\extramarks{Problem \arabic{#1} (continued)}{Problem \arabic{#1} continued on next page\ldots}\nobreak{}
        \stepcounter{#1}
        \nobreak\extramarks{Problem \arabic{#1}}{}\nobreak{}
    }
    
    \setcounter{secnumdepth}{0}
    \newcounter{partCounter}
    \newcounter{homeworkProblemCounter}
    \setcounter{homeworkProblemCounter}{1}
    \nobreak\extramarks{Problem \arabic{homeworkProblemCounter}}{}\nobreak{}
    
    %
    % Homework Problem Environment
    %
    % This environment takes an optional argument. When given, it will adjust the
    % problem counter. This is useful for when the problems given for your
    % assignment aren't sequential. See the last 3 problems of this template for an
    % example.
    %
    \newenvironment{homeworkProblem}[1][-1]{
        \ifnum#1>0
            \setcounter{homeworkProblemCounter}{#1}
        \fi
        \section{Problem \arabic{homeworkProblemCounter}}
        \setcounter{partCounter}{1}
        \enterProblemHeader{homeworkProblemCounter}
    }{
        \exitProblemHeader{homeworkProblemCounter}
    }
    
    %
    % Homework Details
    %   - Title
    %   - Due date
    %   - Class
    %   - Section/Time
    %   - Instructor
    %   - Author
    %
    
    \newcommand{\hmwkTitle}{Homework\ \#1}
    \newcommand{\hmwkDueDate}{August 29, 2017}
    \newcommand{\hmwkClass}{English}
    \newcommand{\hmwkClassTime}{}
    \newcommand{\hmwkClassInstructor}{Professor Joanne Martin}
    \newcommand{\hmwkAuthorName}{\textbf{Bernardo Meurer}}
    
    %
    % Title Page
    %
    
    \title{
        \vspace{2in}
        \textmd{\textbf{\hmwkClass:\ \hmwkTitle}}\\
        \normalsize\vspace{0.1in}\small{Due\ on\ \hmwkDueDate\ at 11:59pm}\\
        \vspace{0.1in}\large{\textit{\hmwkClassInstructor\ \hmwkClassTime}}
        \vspace{3in}
    }
    
    \author{\hmwkAuthorName}
    \date{}
    
    \renewcommand{\part}[1]{\textbf{\large Part \Alph{partCounter}}\stepcounter{partCounter}\\}
    
    %
    % Various Helper Commands
    %
    
    % Useful for algorithms
    \newcommand{\alg}[1]{\textsc{\bfseries \footnotesize #1}}
    
    % For derivatives
    \newcommand{\deriv}[1]{\frac{\mathrm{d}}{\mathrm{d}x} (#1)}
    
    % For partial derivatives
    \newcommand{\pderiv}[2]{\frac{\partial}{\partial #1} (#2)}
    
    % Integral dx
    \newcommand{\dx}{\mathrm{d}x}
    
    % Alias for the Solution section header
    \newcommand{\solution}{\textbf{\large Solution}}
    
    % Probability commands: Expectation, Variance, Covariance, Bias
    \newcommand{\E}{\mathrm{E}}
    \newcommand{\Var}{\mathrm{Var}}
    \newcommand{\Cov}{\mathrm{Cov}}
    \newcommand{\Bias}{\mathrm{Bias}}
\begin{document}
\maketitle
\pagebreak
\begin{homeworkProblem}
	\textbf{What is an essay's thesis? Restate it in your own words.}

	The thesis is the main idea of the essay. It is the core of the
	argumentation that will take place in the essay. In other words,
	the thesis is the \emph{kernel} of the argument.
\end{homeworkProblem}

\begin{homeworkProblem}
    \textbf{List the arguments Herman presents as evidence to support 
    his thesis.}

    \begin{itemize}
        \item ``In a college setting, it is all but impossible not to know a 
        person who is older than 21 and willing to provide alcohol to younger
        students''
        \item ``Each year 1,400 [college students] die from drinking too much.
        600,000 are victims of alcohol-related physical assault and 17,000 are
        a result of drunken driving deaths, many being innocent bystanders''
        \item ``70,000 people, overwhelmingly female, are annually sexually 
        assaulted in alcohol-related situations''
        \item ``For 10- to 24-year-olds, alcohol is the fourth-leading cause of
        death, made so by factors ranging from alcohol poisoning to alcohol
        related assault and murder''
    \end{itemize}
\end{homeworkProblem}

\begin{homeworkProblem}
    \textbf{Summarize the opposing argument the essay identifies. Then, 
    summarize Herman's refutation of the argument.}

    The counter-argument mentioned by the essay is that the U.S. is one of only
    four countries in the \emph{world} forcing drinking age to 21, the idea
    being that \emph{every other country} has a lower drinking age.
    
    Herman's response to this is to argue that, if it is precisely those under
    21 that are abusing alcohol, it makes little sense to \emph{allow} them to
    do it; and that instead the prohibition should be sterner.
\end{homeworkProblem}

\begin{homeworkProblem}
    \textbf{Restate the essay's concluding statement in your own words.}
    We should all worry and think carefully about the alcohol problem, because
    it might very well take our own life or the life of someone we love.
\end{homeworkProblem}

\end{document}