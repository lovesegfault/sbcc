\documentclass[12pt,letterpaper]{article}
\usepackage[utf8]{inputenc}
\usepackage[pass]{geometry}
\usepackage{mla}
\usepackage[hidelinks]{hyperref}

\urlstyle{same}

\begin{document}
\begin{mla}{Bernardo}{Meurer}{Professor Eubank}{WEXP 290C 20168}{\today}{Video Assignment --- What Makes Us Feel Good About our Work?}
    The video I watched presents a fascinating discussion on what intrinsic properties humans value most when performing an activity. One of the experiments, for example, consisted of giving the subjects a piece of paper with some randomized task and asking them to find repeated characters, giving them money for every paper turned in. Then, depending on their control group, their paper would either be analyzed and given feedback, placed straight into a pile, or shreded in front of
    them. What I found most interesting in the above example was the fact that the results for shredding their work and for just ignoring it were basically identical, meaning that to ignore and not value one's work has the same effect on their morale as if you were destroying their work. The video presents other experiments of similar nature, but they all corroborate the final point that our value of work is based both on our perception of others' valuations of it, as well as the
    personal effort put into it.

    I had to idea, before watching the video, that ignoring someone's work could be as demotivating as destroying it, and I feel like I am very much guilty of letting others' efforts go by unnoticed. With this in mind, I will do my best from now on to always provide positive reinforcement to others when they do good work or, in a more general sense, perform an action I appreciate or admire.
\end{mla}
\end{document}


